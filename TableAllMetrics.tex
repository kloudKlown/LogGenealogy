%% LyX 2.1.2 created this file.  For more info, see http://www.lyx.org/.
%% Do not edit unless you really know what you are doing.
\documentclass{article}
\usepackage[latin9]{inputenc}
\usepackage{array}
\usepackage{rotating}
\usepackage{multirow}

\makeatletter

%%%%%%%%%%%%%%%%%%%%%%%%%%%%%% LyX specific LaTeX commands.
%% Because html converters don't know tabularnewline
\providecommand{\tabularnewline}{\\}

\makeatother

\begin{document}
\begin{table*}
\centering \protect\caption{Taxonomy of metrics considered for model construction}


\label{tba:Taxonomy} %
\begin{tabular}{|c|l|c|>{\raggedright}p{1\columnwidth}|}
\hline 
Dimension  & Metrics  & Values  & Definition (d) -- Rationale (r)\tabularnewline
\hline 
\hline 
\multirow{12}{*}{Change Metrics} & \begin{turn}{90}
Log revision count
\end{turn} & Numeric & d: The number of prior commits which had log changes. \tabularnewline
\cline{2-4} 
 &  &  & r: This helps to identify if the file is prone to log changes.\tabularnewline
\cline{2-4} 
 & New File & Boolean (0 -1) & d: Check if the log is added in a new file (i.e., newly committed
)\tabularnewline
\cline{2-4} 
 &  &  & r: This helps to identify which logs where added later in subsequent
commits.\tabularnewline
\cline{2-4} 
 & Total revision count & Numeric & d: Total number of commits made to the file before the log is added.
This value is 0 for logs added in the initial commit but not for logs
added overtime. \tabularnewline
\cline{2-4} 
 &  &  & r: This helps to find out if the file is changed heavily which can
result in log changes~\cite{Characterizinglogs}. \tabularnewline
\cline{2-4} 
 & Code churn in commit & Numeric & d: The code churn of the commit in which a log is added. \tabularnewline
\cline{2-4} 
 &  &  & r: Log changes are correlated to code churn in files~\cite{EMSEIAN}. \tabularnewline
\cline{2-4} 
 & Variables declared & Numeric & d: The number of variables which are declared before the log statement.
(we limit to 20 lines before log statement).\tabularnewline
\cline{2-4} 
 &  &  & r: When new variables are declared, developers may log the new variables
to obtain more information~\cite{Characterizinglogs}. \tabularnewline
\cline{2-4} 
 & SLOC & Numeric & d: The number of lines of code in the file.\tabularnewline
\cline{2-4} 
 &  &  & r: Large files have more functionality and are more prone to changes~\cite{zhang2009investigation}
and more log changes~\cite{Characterizinglogs,EMSEIAN}. \tabularnewline
\hline 
\multirow{22}{*}{Product Metrics} & Log context & Categorical & d: Identify the block in which a log is added. (i.e., `if', `if-=else',
`try-catch', `exception', `throw', `new function').\tabularnewline
\cline{2-4} 
 &  &  & r: Prior research finds that logs are mostly used in assertion checks,
logical branching, return value checking, assertion checking~\cite{Fu1}. \tabularnewline
\cline{2-4} 
 & Log variable count & Numerical & d: Number of variables logged.\tabularnewline
\cline{2-4} 
 &  &  & r: Over 62\% of logs add new variables~\cite{Characterizinglogs}.
Hence fewer variables in the initial log statement might result in
addition later. \tabularnewline
\cline{2-4} 
 & Log density & Numerical & d: Ratio of number of log lines to the source code lines in the file.\tabularnewline
\cline{2-4} 
 &  &  & r: Research has found that there is on average one log line per 30
lines of code~\cite{Characterizinglogs}. If it is less it suggests
there may be additions in later commits.\tabularnewline
\cline{2-4} 
 & Log level & Categorical & d: Identify the log level (verbosity) of the added log (i.e., `info',
`error', `warn', `debug', `trace' and `trace').\tabularnewline
\cline{2-4} 
 &  &  & r: Developers spend significant amount of time in adjusting the verbosity
of logs~\cite{Characterizinglogs}. \tabularnewline
\cline{2-4} 
 & Log text length & Numerical & d: Number of text phrases logged (i.e., we count all text present
between a pair of quotes as one phrase).\tabularnewline
\cline{2-4} 
 &  &  & r: Over 45\% of logs have modifications to static context~\cite{Characterizinglogs}.
Logs with fewer phrases might be subject to changes later to provide
better explanation.\tabularnewline
\cline{2-4} 
 & Resolution time & Numerical & d: The time it takes for the issue to get fixed. It is defined as
the time it takes since an issue is opened until closed. \tabularnewline
\cline{2-4} 
 &  &  & r: More resolution time might suggest a more complex fix with more
log churn. \tabularnewline
\cline{2-4} 
 & Number of developers involved & Numerical & d: Total number of unique developers who comment on the issue report
on JIRA.\tabularnewline
\cline{2-4} 
 &  &  & r: Components with many unique authors likely lack strong ownership,
which in turn may lead to more defects~\cite{mcintosh2014impact}
and change logs~\cite{EMSEIAN}. \tabularnewline
\cline{2-4} 
 & Number of comments & Numericl & d: Total number of discussion posts on the issue. \tabularnewline
\cline{2-4} 
 &  &  & r: Number of comments is correlated to the resolution time of issue
reports~\cite{giger2010predicting}. More comments may also indicate
the issue is more complex resulting in more code churn and log changes. \tabularnewline
\cline{2-4} 
 & Developer experience & Numericl & d: The number of commits the developer has made prior to this commit. \tabularnewline
\cline{2-4} 
 &  &  & r: Research has shown that experienced developers might take up more
complex issues~\cite{rahman2011ownership} and therefore may leverage
logs more~\cite{EMSEIAN}. \tabularnewline
\cline{2-4} 
 & Issue type  & Categorical & d: Identify the type of issue, i.e., `Bug', `Improvement', `Task',
`New Feature', `Sub-Task', `Test'. \tabularnewline
\cline{2-4} 
 &  &  & r: Some issue types might have higher code churn than others (example:
Bug and New features might have more code churn when compared to Sub-Tasks)
and are committed faster.\tabularnewline
\cline{2-4} 
 & Priority type & Categorical & d: Identify the priority of the issue i.e., `Critical', `Blocker',
`Major', `Minor' and `Trivial'\tabularnewline
\cline{2-4} 
 &  &  & r: Research has shown that priority of issue affects resolution time
of bug fixes~\cite{MarkFixTime}. A Higher priority indicates the
issue will be fixed faster with log changes. \tabularnewline
\hline 
\end{tabular}\protect
\end{table*}

\end{document}
