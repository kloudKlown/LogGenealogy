 %%%%%%%%%%%%%%%%%%%%%%% file template.tex %%%%%%%%%%%%%%%%%%%%%%%%%
%
% This is a general template file for the LaTeX package SVJour3
% for Springer journals.          Springer Heidelberg 2010/09/16
%
% Copy it to a new file with a new name and use it as the basis
% for your article. Delete % signs as needed.
%
% This template includes a few options for different layouts and
% content for various journals. Please consult a previous issue of
% your journal as needed.
%
%%%%%%%%%%%%%%%%%%%%%%%%%%%%%%%%%%%%%%%%%%%%%%%%%%%%%%%%%%%%%%%%%%%
%
% First comes an example EPS file -- just ignore it and
% proceed on the \documentclass line
%% your LaTeX will extract the file if required
%\begin{filecontents*}{example.eps}
%%!PS-Adobe-3.0 EPSF-3.0
%%%BoundingBox: 19 19 221 221
%%%CreationDate: Mon Sep 29 1997
%%%Creator: programmed by hand (JK)
%%%EndComments
%
%\end{filecontents*}
%
\RequirePackage{fix-cm}
%
%\documentclass{svjour3}                     % onecolumn (standard format)
%\documentclass[smallcondensed]{svjour3}     % onecolumn (ditto)
%\documentclass[smallextended]{svjour3}       % onecolumn (second format)
%\documentclass[twocolumn]{svjour3}          % twocolumn
%
%\smartqed  % flush right qed marks, e.g. at end of proof
%
\documentclass[conference]{IEEEtran}
\input{setup}
\usepackage{listings}
\usepackage{verbatimbox}
%\usepackage{xspace}
\usepackage{setspace}

\usepackage{graphicx}
%
% \usepackage{mathptmx}      % use Times fonts if available on your TeX system
%
% insert here the call for the packages your document requires
%\usepackage{latexsym}
% etc.
%
% please place your own definitions here and don't use \def but
% \newcommand{}{}
%
% Insert the name of "your journal" with
% \journalname{myjournal}
%
\lstset{ %
 % basicstyle=\scriptsize,        % the size of the fonts that are used for the code
 basicstyle=\singlespacing\ttfamily\scriptsize,
  breaklines=true,                 % sets automatic line breaking
  captionpos=b,                    % sets the caption-position to bottom
  frame=single,	                   % adds a frame around the code
  keepspaces=true,                 % keeps spaces in text, useful for keeping indentation of code (possibly needs columns=flexible)
  keywordstyle=\color{black},       % keyword style
  language=make,                 % the language of the code
  numbers=left,                    % where to put the line-numbers; possible values are (none, left, right)
  numbersep=5pt,                   % how far the line-numbers are from the code
  numberstyle=\tiny\color{mygray}, % the style that is used for the line-numbers
  rulecolor=\color{black},         % if not set, the frame-color may be changed on line-breaks within not-black text (e.g. comments (green here))
  showspaces=false,                % show spaces everywhere adding particular underscores; it overrides 'showstringspaces'
  showstringspaces=false,          % underline spaces within strings only
  showtabs=false,                  % show tabs within strings adding particular underscores
  stringstyle=\color{black},     % string literal style
  tabsize=2,	                   % sets default tabsize to 2 spaces
  title=\lstname,                   % show the filename of files included with \lstinputlisting; also try caption instead of title
  firstnumber=1
}
\newfloat{Code}{H}{myc}

\newcommand{\cp}[1]{\textcolor{red}{{\it [CP: #1]}}}


\begin{document}

\title{Examining the Stability of Logging Statements}
 %\thanks{Grants or other notes
%about the article that should go on the front page should be
%placed here. General acknowledgments should be placed at the end of the article.}

%\subtitle{Do you have a subtitle?\\ If so, write it here}

%\titlerunning{Short form of title}        % if too long for running head

%\author{Suhas Kabinna         \and
%	Weiyi Shang \and
%	Cor-Paul Bezemer \and
%	Ahmed E. Hassan
%}
\author{\IEEEauthorblockN{Suhas Kabinna, 	Cor-Paul Bezemer and Ahmed E. Hassan}
\IEEEauthorblockA{	Software Analysis and Intelligence Lab (SAIL) \\
	Queen's University \\
	Kingston, Ontario \\
	Email:{\{kabinna, bezemer, ahmed\}@cs.queensu.ca} \\ \\}
\and
\IEEEauthorblockN{Weiyi Shang }
\IEEEauthorblockA{	Department of Computer Science and Software Engineering \\
		Concordia University\\
		Montreal, Quebec \\
		Email: {shang@encs.concordia.ca}
}
}
%\authorrunning{Short form of author list} % if too long for running head
%
%\institute{Suhas. Kabinna,  Cor-Paul. Bezemer and Ahmed E.Hassan \at
%	Software Analysis and Intelligence Lab (SAIL) \\
%	Queen's University \\
%	Kingston, Ontario \\
%	\email{ \{kabinna,bezemer,ahmed\}@cs.queensu.ca }  \\
%	Weiyi Shang \at
%	Department of Computer Science and Software Engineering \\
%	Concordia University\\
%	Montreal, Quebec \\
%	\email{ shang@encs.concordia.ca}  
%	%             \emph{Present address:} of F. Author  %  if needed
%}

\date{Received: date / Accepted: date}
% The correct dates will be entered by the editor


\maketitle

\begin{abstract}

%Logs are system generated outputs, created by logging statements in the code. Logs assist in understanding system behavior, monitoring choke-points and debugging. Prior research has demonstrated the importance of logs in operating, understanding and improving software systems. The importance of logs has lead to a new market of log management applications and tools. However, logs are often unstable i.e., being changed without the consideration of other stakeholders, causing misleading results and failure of log analysis tools. In order to pro actively mitigate such issues that are caused by unstable logs, in this paper we empirically study the stability of logs in four large software applications namely: Liferay, ActiveMQ, Camel and CloudStack. We find that 45\%-55\% of the logs are changed in these software applications. We build a random forest classifier to model if a log added to a file will be changed (i.e., effect the static text, variables or verbosity level of the logs) in the future, using context and log metrics calculated at the time of introduction of the log. Our classifier can model which logs will be changed in the future with 89\%-91\% precision and 71\%-83\% recall. We find that file ownership, developer experience, log density and SLOC are strong predictors of log stability in our models. Our findings can help develop more robust log processing tools and also help system administrators identify which logs are more likely to change and avoid depending on such unstable logs through critical analysis. 
Logging statements produce logs that assist in understanding system behavior, monitoring choke-points and debugging. Prior research demonstrated the importance of logging statements in operating, understanding and improving software systems. The importance of logs has lead to a new market of log management and processing tools. However, logs are often unstable, i.e., the logging statements that generate logs are often changed without the consideration of other stakeholders, causing misleading results and failures of log processing tools. In order to proactively mitigate such issues that are caused by unstable logging statements, in this paper we empirically study the stability of logging statements in four open source applications namely: Liferay, ActiveMQ, Camel and CloudStack. We find that 20-45\% of the logging statements in our studied applications change throughout their lifetime. The median number of days between the introduction of a logging statement and the first change to that statement is between 1 and 17 in our studied applications. These numbers show that in order to reduce maintenance effort, developers of log processing tools must be careful when selecting the logging statements on which they will let their tools depend.

In this paper, we make an important first step towards assisting developers of log processing tools in determining whether a logging statement is likely to remain unchanged in the future. Using random forest classifiers, we examine which metrics are important for understanding whether a logging statement will change. 
%We use metrics that are calculated from context and log information, to build a random forest classifier to understand which factors can increases the likelihood of a log change. 
We show that our classifiers achieve 83\%-91\% precision and 65\%-85\% recall in the four studied applications. We find that file ownership, developer experience, log density and file-size are important metrics for determining whether a logging statement will change in the future. Developers can use this knowledge to build more robust log processing tools, by making those tools depend on logs that are generated by logging sCtatements that are likely to remain unchanged.

% By understanding which metrics are important for deciding whether a logging statement, practitioners can avoid the logs which have a higher likelihood of being changed and develop robust log processing tools.


%Our findings can help practitioners avoid depending on such unstable logs through critical analysis and develop more robust log processing tools

%which logs are more likely to change


% However, logs may change over time due to debugging, improvement or addition of new features and these changes have to be communicated to operators and administrators. In this paper, we study the different factors which can affect log stability. We conduct a case study on four large software applications namely: Liferay, ActiveMQ, Camel and CloudStack. We find that 45\%-55\% of the logs are changed in these software applications. We identify the log changes which effect the static text, variables or verbosity level of log and exclude other changes to logs. Next, we build models to predict if a log added to a file will change in the future. We use context and log  metrics calculated at the time of introduction of the log, to build a random forest classifier. Our classifier can predict which logs will change in the future with 89\%-91\% precision and 71\%-83\% recall. We find that file ownership, developer experience, log density and SLOC are strong predictors of log stability in our models and can help identify which logs are more likely to change in the future.On the one hand, this can help developers of log processing tools to develop more robust applications. On the other hand, system administrators can know before hand which logs are more likely to cause issues in the log processing tools, which can reduce their maintenance costs.


% We identify the four possible types of log changes namely 1) text modification, 2) variable modification, 3) log level change and 4) log relocations. We find that log relocation changes have no affect on log processing tools and we exclude these from our analysis.


% On the one hand, this can help developers of log processing tools to develop more robust applications. On the other hand, system administrators can know before hand which logs are more likely to cause issues in the log processing tools, which can reduce their maintenance costs.

%\keywords{First keyword \and Second keyword \and More}
% \PACS{PACS code1 \and PACS code2 \and more}
% \subclass{MSC code1 \and MSC code2 \and more}
\end{abstract}

\section{Introduction}
\label{intro}
Logs are leveraged by developers to record useful information during the execution of a system. Loinggs are recorded during various developmental activities such as bug fixing~\cite{ConsoleLogs,JGLouMining,QFuanomaly}, performing improvement tasks~\cite{Automatic}, monitoring performance~\cite{Yuan} and for knowledge transfer~\cite{IanWCRE}.
Logging can be done through use of log libraries or archaic methods such as \textsl{print} statements. Every log contains a textual part, which provides information about context, a variable part providing information about the event and a log level, which shows the verbosity of the logs. An example of a log is shown below where info is the logging level, \textsl{Testing Connection to Host Id} to is the event and the value of variable \textsl{host} is the variable part.
\hypobox{LOG.info( ``Testing Connection to Host Id:" + host);}

The rich  and unified format of logs has lead to the development of many log processing tools such as \textsl{Splunk}, \textsl{Xpolog}, \textsl{Logstash} and in-house tools. These log processing tools are used to generate information for capacity planning of large-scale systems~\cite{hassan2008industrial,nagappan2009efficiently}, to monitor system health~\cite{bitincka2010optimizing} or to detect abnormal system behavior~\cite{JiangICSM2008}. These applications rely heavily on the log messages themselves and require continuous maintenance when the format or content of logs are changed. 

%These tools are used for a variety of purposes and not only by developers for developmental activities. They are used to monitor system health, to detect anomalies, to find performance issues and also capacity planning. Because of this new area of log utilization there have been many commercial log processing tools like \textsl{Splunk}, \textsl{Xpolog}, \textsl{Logstash} and in-house tools. These applications reply heavily on the log messages themselves and require continuous maintenance when the format or content of logs are changed. 

Research shows that only 40\% of the logs at execution level, stays the same across releases and the impact of 15-80\% of the changes can be minimized through robust analysis~\cite{IanWCRE}. These log changes, can affect the log processing tools which heavily depend on them and maintenance cost will be high. In this paper, we track the changes made to logs across multiple releases in the four studied systems. In order to get a better understanding of the log changes we focus our research on the following RQ's.

% and use the data to answer the following research questions.

\textbf{RQ1:} \textbf{How much do logs change over time and why do the changes occur?}

Based on our quantitative analysis of the studied systems we identify three categories of change frequency in logs. If a log is changed more than four times we categorize it as \textsl{`Frequently Changed'}, if it has only three changes or less  it is categorized as \textsl{`Changed'} and if there are no changes made we categorize it as \textsl{`Never Changed'}. We find that 20-80\% of all logs are changed at least once throughout the lifespan of our studied systems. We find developers change logs for four main reasons namely:`change of log level', `text modification', `variable modification' and `log relocation'.
% We find that only \textsl{Hadoop} has 60\% of log statements in the \textsl{`Never Changed'} category.In all other projects we find that \textsl{`Changed'} is majority ranging between 20\% - 68\% in CloudStack. We find that \textsl{`Frequently changed'} ranges between 1\% - 13\% in the four subject systems and is the least among the three categories. These results show that logs are not stable and there is need to study the stability of log statements in large subject systems.


\textbf{RQ2:} \textbf{Can code, log and developer metrics help in explaining the stability of logs?}

 We find that code, log and developer related metrics help in building models which can predict which logs are more likely to change in the future. We use the data from three dimensions namely, code, log and developers. Our \textsl{random forest} achieved an accuracy of 89\% to 93\% in all studied systems with recall of 76\% to 92\%, when predicting which logs have higher likelihood of getting changed. We also identify significant metrics from each dimension, that affect the stability of logs. We find several metrics (e.g., developer experience, source lines of code, \# of comments, log text length) are strong predictors in predicting if a log will change in the future. 
 
 
 These results show that code, log and developer related metrics can help in identifying unstable logs in our studied systems. This can help in reducing the effort needed in the maintenance of log processing applications, as system maintainers can flag the logs that have the potential of being changed in subsequent releases and track them. 
 
The rest of this paper is organized as follows. Section~\ref{Methodology} presents the methodology for gathering and extracting data for our study. Section~\ref{studyresults} presents the case studies and the results to answer the two research questions. Section~\ref{related} describes the prior research that is related to our work. Section~\ref{threats} discusses the threats to validity. Finally, Section~\ref{conc} concludes the paper.
 
 
% However, as these tools are not scalable for all companies and systems, companies prefer in house development or customization of these tools for their specific purposes. 


%\section{Methodology}
%\label{Methodology}
%In this section we present our rationale for selecting the systems we studied and present our data extraction and analysis approach.


In this paper, we aim to find the unstable logs in a system for easier management of log processing tools. We cloned over 15 projects which had more than 10,000 commits from git repository locally. We also verify if the projects utilize a bug tracking system like 'JIRA' or `Bugzilla' because, this helps to tag commits to specific development activities (i.e., bug fix, improvement, new-features). We use the `grep' command to recursively search for all log lines within `.java' files in each project in the cloned repositories. We picked the top four projects with the highest log line count. The four open source projects are: Liferay, Camel, ActiveMQ and CloudStack. All studied systems have extensive system logs and Table~\ref{tba:overviewsystems} presents an overview of the systems.

\begin{table}[tbh]
\centering \protect\caption{An overview of all studied systems}


\label{tba:overviewsystems} %
\begin{tabular}{lllll}
\hline 
Projects  & Liferay  & Camel  & ActiveMQ  & CloudStack \tabularnewline
\hline 
Starting release  & 6.1.0-b3  & 1.6.0  & 4.1.1  & 2.1.3 \tabularnewline
End release  &7.0.0-m3  & 2.11.3  & 5.9.0  & 4.2.0 \tabularnewline
Total \# of releases   & 24  & 43  & 19 & 111 \tabularnewline
Total added code  & 3.9M  & 505k  & 261k  & 1.09M \tabularnewline
Total deleted code  & 2.8M  & 174k  & 114k  & 750K \tabularnewline
Total \# added logs  & 10.4k  & 5.1k  & 4.5k  & 24k \tabularnewline
Total \# deleted logs  & 8.1k  & 2.4k  & 2.3k  & 17k \tabularnewline
\hline 
\end{tabular}
\end{table}

\textbf{Liferay}\footnote[1]{http://www.liferay.com/}:  Liferay is a free and open source enterprise project written in Java. It provides platform features which are used in development of websites and portals.
% We select liferay as it is one of the leading platforms for platform development~\cite{LiferayGartner} and has growing user base~\cite{LiferayUser}.
  It also has an extensive issue tracking system in JIRA which helps in categorizing the commits as bug fixes, improvements, . We study the releases from 6.1.0 to 7.0.0-m3 which cover more than three years of development from 2010 to 2014

%Hadoop is an open source software framework for distributed storage and for the processing of big data on clusters. Hadoop uses the MapReduce data-processing paradigm. The logging characteristics of Hadoop have been studied in prior research~\cite{IanWCRE,EMSEIAN,IanContextinformation}. We study the releases from Hadoop 0.20.1 to 2.2.0.


\textbf{Camel}\footnote[2]{http://camel.apache.org/}: Camel is an open source integration platform based on enterprise integration patterns. We analyze Camel release 1.6 to 2.11.3 which cover more than five years of development from 2009 to 2013. 
\begin{figure}[tb]
	\centering
	\includegraphics[width=1\linewidth,height=0.29\textwidth]{LogGenalogyMethdology}
	\caption{Overview of the data extraction and case study approach}
	\label{fig:LGmethod}
\end{figure}



\textbf{ActiveMQ}\footnote[3]{http://activemq.apache.org/}: ActiveMQ is an open source message broker and integration patterns server. We covered ActiveMQ release 4.1.1 to 5.9.0 which cover more than 6 years of development from 2007 to 2013.


\textbf{CloudStack}\footnote[4]{https://cloudstack.apache.org/}: Apache CloudStack is open source software designed to deploy and manage large networks of virtual machines, as a highly available, highly scalable Infrastructure-as-a-Service (IaaS) cloud computing platform. We covered CloudStack release 2.1.3 to 4.20 which cover more than 3 year of development from 2010 to 2014.

In the remainder of this section we present our approach for preparing the data to answer
our research questions.


\subsection{Extracting Code Evolution}
In order to find the stability of logs we have to identify all the `Java' files in our studied systems. To achieve this, we clone the \emph{master} branch of the git repository each studied system locally. We use the `find' command to recursively find all the files which end with pattern `*.java'. To remove the \textsl{Java Test} files, we use `grep' command to filter all files which have `test' or `Test' in their pathname.

 After collecting all the Java files from the four studied systems, we use their git repository to obtain all the changes made to the file within the time-frame discussed above. We use the `follow' option to track the file even when they are renamed or relocated within our studied systems. We flatten all the changes made to files to include the changes made in different branches but exclude the final merging commit. Using this approach, we obtain a complete history of each \emph{Java} file. 

\subsection{Identifying Log Changes}
From the extracted code evolution data for each \emph{Java} file, we identify all the log changes made in the commits. To identify the log statements in the source code, we manually sample some commits from each studied system and identify the logging library used to generate the logs. We find that the studied systems use \textsl{Log4j}\footnote{http://logging.apache.org/log4j/2.x/}~\cite{EMSEIAN} and \textsl{Slf4j}\footnote{http://www.slf4j.org/} widely and \textsl{logback}\footnote{http://logback.qos.ch/} sparingly. Using this information we identify the common method invocations that invoke the logging library. For example in  ActiveMQ and Camel a logging library is invoked by method named `LOG' as shown below.
\hypobox{ LOG.debug(``Exception detail", exception);}

In CloudStack it is usually done through `\_logger' as follows.

\hypobox{\_logger.warn(``Timed out: " + buildCommandLine(command));}

As projects can have multiple logging libraries throughout its life-cycle, we use regular expressions to match all the common log invocation patterns (i.e., `LOG',`log',`\_logger',`LOGGER',`Log'). We count every such invocation of a logging library as a log statement.


\subsection{Tracking Log Changes}
After identifying all the log changes made to a file across multiple commits, we track each log individually to find out whether it has changed in subsequent revisions. To track the log changes made, we used the Levenshtein ratio~\cite{levenshteinratio}. We first collect all the logs present in a file at the first commit, which form the initial set of log statements for the file. Every subsequent commit which has changes to a log appears as an added and deleted log statement in \textsl{git}.  

 We consider a pair of added and deleted logs as a modification if they have a Levenshtein ratio of 0.6 or higher. We arrive at this threshold by following a iterative process of manually analyzing the log pairs. 
 
 To calculate Levenshtein ratio for a pair of added and deleted logs, we first remove the logging method (e.g., LOG) and compare the remaining text. For example in the logs shown below, we remove the logging method (i.e., LOG) and find that the Levenshtein ratio between the added and deleted pair is 0.86. Hence, this log change is categorized as a log modification.  
 \hypobox{+      LOG.debug(``Call: " +method.getName()+`` took "+ callTime + ``ms");\\ 
 	-        LOG.debug(``Call: " +method.getName()+ `` " + callTime);} 


If the added log does not match any log in our initial set, it is considered as a new log addition and is added to the initial set for tracking in future commits. From this we can track how many times a log is changed and how many commits are made between the changes. 


\subsection{Match JIRA to Log Changes}

Using the commit data, we also track the JIRA issues to extract the developer metrics. To achieve this, we extract the JIRA issue IDs from commit messages and use the JIRA repository to extract the JIRA issue. The JIRA issue contains information about the issue such as, type (i.e., bug, improvement, new-feature, task), priority, resolution time and developer information such as \# of comments and \# of developers involved in the JIRA discussion. We use this information along with code and log churn metrics for answering our research questions. 




 









\section{Preliminary analysis}

\label{analysis}
%In this section, we present our study results by answering our research questions. For each question, we discuss the motivation behind it, the approach to answering it and finally the results obtained. 
%\\
%
%\noindent\textbf{RQ1:} \textbf{How much do logs change over time and why do the changes occur?}
%\\

%\noindent\textbf{Motivation}

%Research has shown that logs evolve along with the code~\cite{IanContextinformation}. When logs are changed, the log  processing tools which are dependent on them also have to get updated. This results in costly maintenance effort. To understand the cost, we have to understand how frequent changes to logs are. Hence, we explore the frequency of changes to logs in our studied systems and why they are changed.\\

%\noindent \textbf{Approach}


%\begin{table*}
%	\centering
%	\caption{Distribution of log changes in different projects}
%	\label{tba:logchangeDistribution}
%	\begin{tabular}{l|>{\centering}p{.1\columnwidth}>{\centering}p{.01\columnwidth} 
%			p{0.01\columnwidth} }
%		\cline{1-4}  	\multicolumn{1}{|c}{Projects}    & \multicolumn{1}{|c}{Never Changed (\%) }  &  \multicolumn{1}{|c}{Changed (\%) }	   &  \multicolumn{1}{|c}{Frequently Changed (\%) }\\ \cline{1-4}   
%		
%		Life Ray      & 78.67     & 19.66 & 1.66           \\
%		
%		Camel      & 55.43    & 37.32 & 7.25            \\
%		ActiveMq   & 34.78     & 62.02 & 3.20           \\
%		CloudStack & 19.68     & 68.61 & 11.71          \\ \cline{1-4}
%	\end{tabular}
%\end{table*}

%\begin{table*}
%	\centering
%	\caption{Distribution of log changes in different projects}
%	\label{tba:logtype}
%	\begin{tabular}{l|llll}
%		\cline{1-5}  	\multicolumn{1}{|c}{Projects}    & \multicolumn{1}{|c}{ ActiveMQ }  &  \multicolumn{1}{|c}{ Camel}	   &  \multicolumn{1}{|c}{ Cloudstack }  & 
%		\multicolumn{1}{|c|}{ Liferay } \\ \cline{1-5}   
%		
%		Log relocation (\%)       & 20.05     & 20.80 &  53.32  & 49.42         \\
%		
%		Log text change (\%)      & 26.59    & 26.37 & 15.54    & 32.70       \\
%		Log variable change (\%)   & 53.18     & 52.75 & 31.08 &  16.75     \\
%		Change of log level (\%) & 0.18   & 0.08 & 0.04  &  1.13       \\ 
%%		Text and variable change (\%) & 2.33     & 6.39 & 22.0   &  30.59    \\ 
%\cline{1-5}
%	\end{tabular}
%\end{table*}


%In this paper we aim to get a better understanding of the
%unstable logs in an application so that we can improve the maintenance of log processing tools. 
In this paper we study the changes that are made to logs across multiple releases in open source projects. The goal of our study is to present a model for deciding whether a logging statement is likely to change in the future. This model can assist developers of log processing tools to decide on which logging statements they want their tool to depend. %In order to understand and build a random forest classifier to predict log changes, it is necessary to identify the extent logs are changed within these applications.
First, we perform a preliminary analysis, in which we examine how often logging statements change, to motivate our work. In this section, we present our rationale for selecting the projects that we studied and present the results of our preliminary analysis of the four studied projects. 

\subsection{Studied Projects}
In this paper, we examine the logging statements in open source projects. We selected these projects based on the following three criteria:
\begin{itemize}
	\item \textbf{Log usage} - The selected projects must make extensive use of logs in their source code
	%	This helps to improve the performace of the random forest classifier and to identify the factors which effect log stability.
	\item \textbf{Project activity} - The projects must have a mature development history (i.e., more than 10,000 commits)
	\item \textbf{Technology used} - To simplify the implementation of our study, we opted to select only projects that are written in Java and are available on a Git repository
%	We pick applications written in Java as it is one of the most popular languages today~\cite{Javaprog}.
%	 to identify and track the log changes across multiple releases. 
\end{itemize}

To select projects matching these criteria, we first selected all Java projects from the list of Apache Foundation Git repositories\footnote{\url{https://git.apache.org/}} that have more than 10,000 commits. Next, we counted the number of logging statements in a repository using the \code{grep} command in Listing~\ref{lst:grep}.

\vspace{-5mm}
\begin{Code}
\begin{lstlisting}[caption={Counting logging statements}, label={lst:grep}]
grep -icR 
"\(log\|s\_logger\)\.\(warn\|info\|error\|debug\|warn\)(" . 
| grep "\.java"
\end{lstlisting}
\end{Code}
\vspace{-5mm}




%To find the log usage in an project we use the \emph{grep} command to search all lines of code within the \emph{.java} files. %Next, using \emph{git log} we find the total number of commits in the code repositories and select projects which have more than 10,000 commits. 

We selected the four projects (ActiveMQ, Camel, Cloudstack and Liferay) with the highest number of logging statements for further analysis. ActiveMQ\footnote{\url{http://activemq.apache.org/}} is an open source message broker and integration patterns server. Camel\footnote{\url{http://camel.apache.org/}} is an open source integration platform based on enterprise integration patterns. CloudStack\footnote{\url{https://cloudstack.apache.org/}} is an open source project designed to deploy and manage large networks of virtual machines. Liferay\footnote{\url{http://www.liferay.com/}} is an open source platform for building websites and web portals. Table~\ref{tba:overviewsystems} presents an overview of the studied projects. %We pick the releases after incubation for each project, as during incubation the projects might not be used by log processing tools.  


\begin{table}[tb]
	\centering \protect\protect\caption{An overview of the studied projects}
	\smaller
	
	\label{tba:overviewsystems} %
	\begin{tabular}{lrrrr}
		\toprule 
		Projects  & ActiveMQ  & Camel  & CloudStack  & Liferay \\
		\midrule
		\# of logging statements  & 5.1k  & 6.1k  & 9.6k  & 1.8k \\
		\# of commits &	 ?  & ?  & ?  & ? \\	
		\# of months in repository &	 ?  & ?  & ?  & ? \\
%		\# of releases  & 19  & 43  & 111  & 24 \\
		\midrule
		\# of added lines of code  & 261k  & 505k  & 1.09M  & 3.9M \\
		\# of deleted lines of code  & 114k  & 174k  & 750k  & 2.8M \\
		\# of added logging statements  & 4.5k  & 5.1k  & 24k  & 10.4k \\
		\# of deleted logging statements  & 2.3k & 2.4k  & 17k  & 8.1k \\
		\% of logging-related changes & 1.8 & 1.1 & 2.3 & 0.3 \\
		\bottomrule 
	\end{tabular}
\end{table}

\subsection{Data Extraction Approach}

The data extraction approach from the four studied applications consists of three steps: (1) We clone the Git repository of each studied project to extract the change history made for each file (2) We identify the logs present in the file, and using the change history we identify the log changes made in each file (3) We track the log changes that are made to each log in a file across the commits. We use R~\cite{ihaka1996r}, to perform experiments and answer our preliminary analysis and empirical study. Figure~\ref{fig:LGmethod} shows a general overview of our approach and we discuss each step discussed above in further detail. 

\subsubsection*{B.1. Extracting the change history} 
In order to find the stability of logs, we have to identify all the Java files in our studied projects. To achieve this, we use the \emph{grep} command to search for all the \emph{*.java} files in the cloned repositories and we exclude the \emph{test} files. 

After collecting all the Java files from the four studied projects, we use their Git repositories to obtain all the changes that are made to the files within the time-frame shown in Table~\ref{tba:overviewsystems}. We use the \emph{follow} option to track a file even when it is renamed or relocated. We exclude log changes that are made in non-merged branches as they are unlikely to affect log processing tools. We use the \emph{--no-merges} option to flatten the changes to a file and exclude the final merging commit. Using this approach, we obtain a complete history of each Java file in the latest version of the master branch.


\subsubsection*{B.2. Identifying log changes}
From the extracted change history of each {Java} file, we identify all the log changes made in the commits. To identify the log statements in the source code, we manually sample some commits from each studied project and identify the logging library used to generate the logs. We find that the studied projects use \textsl{Log4j}~\cite{EMSEIAN} and \textsl{Slf4j}\footnote{http://www.slf4j.org/} widely and \textsl{logback}\footnote{http://logback.qos.ch/} sparingly. Using this information, we identify the common method invocations that invoke the logging library. For example, in  ActiveMQ and Camel a logging library is invoked by method named \emph{LOG} as shown below.
\hypobox{ LOG.debug(``Exception detail", exception);}


As a project can have multiple logging libraries throughout its life-cycle, we use regular expressions to match all the common log invocation patterns (i.e., \emph{LOG,log,\_logger,LOGGER,Log}). We consider every invocation of a logging library followed by a logging level (\emph{info, trace, debug, error, warn}) a log.


\subsubsection*{B.3. Tracking log changes}
After identifying all the log changes that are made to a file across multiple commits, we track each log individually to find out whether it has changed in subsequent revisions. We first collect all the logs present in a file at the first commit, which form the initial set of logs for the file. Every change to a log in the subsequent commits appears as an added and deleted log in Git. To identify added, deleted and modified logs, we leverage the Levenshtein ratio~\cite{levenshteinratio}. We use Levenshtein ratio instead of string comparison, because Levenshtein ratio quantifies the difference between the strings compared within the range 0 to 1 (more similar the strings the ratio approaches 1). This is necessary to compare multiple logs which can be similar, which is not possible using string comparison.

%within the range 0 to 1 (more similar the strings the ratio approaches 1).

%To calculate Levenshtein ratio for a pair of added and deleted logs, we first remove the logging method (e.g., LOG) and compare the remaining text to increase the accuracy of categorization. 


%We set a minimum threshold of 0.6 for a pair of added and deleted logs to be considered modified. We set 0.6 because there is atleast 60\% similarity between the pair. We find that when the threshold is set lower there are more false positives and when set higher we miss log modifications. When an added log has levenshtein ratio higher than 0.6 with multiple deleted logs, we consider the pair with highest levenshtein ratio. 
To identify log modification we calculate the Levenshtein ratio between all the deleted log and added logs and select the pair which has the highest Levenshtein ratio. This is done recursively to find all the modifications within a commit. For example in the logs shown below, we find that the Levenshtein ratio between the added and deleted pair (a1) is 0.86 and (a2) 0.76. Hence, we consider (a1) as a log modification and compare (a2) with next deleted log. If there are no more deleted log pairs, (a2) is considered as addition of new log into the file. 
\hypobox{-        LOG.debug(``Call: " +method.getName()+ `` " + callTime);\\
	+      LOG.debug(``Call: " +method.getName()+`` took "+ callTime + ``ms"); \space  \space  \space  \space  \space  \space -- \textbf{(a1)}\\ 
	+      LOG.debug(``Call: " +method.setName()+`` took "+ callTime + ``ms");\space  \space  \space  \space  \space  \space -- \textbf{(a2)}} 

This way we track when a log is added into a file and the log is added to the initial set for tracking in future commits. From this, we track how many times a log is changed and how many commits are made between the changes. 

As we cannot identify changes to a log which is added near the end of the time frame, we have to exclude such logs from our analysis. We find that in the studied projects, a new log added changes within 390 commits of being added, as seen from Table~\ref{tba:summaryofnewLogchange}. We exclude all logs added into the project 390 commits before the last commit of our analysis. 


%, we find that this varies widely within the project between 37 to 390 commits. To eliminate such logs, we use the maximum number of commits before a newly added log is changed within the studied projects and exclude all new logs added before that many commits from our analysis.  



%\subsubsection{Match JIRA to Log Changes}

%\subsection{Metric analysis}
%After collecting all the log changes, we look at the different types of log changed which occur in our studied projects. We find that log relocation occurs more often than other types of log changes in all of the studied projects. We find that log relocation occurs between 20\%-50\% as seen in Table~\ref{tba:logtype}. As log relocations have no changes to the text or variables in logs, their effect on log processing tools is limited. Hence we exclude log relocation changes from our datasets and non-relocation changes for the random forest classifier. 




%To achieve this, we clone the \emph{master} branch of the git repository of each studied project locally. We use the `find' command to recursively find all the files which end with pattern `*.java'. To remove the \textsl{Java Test} files, we use the `grep' command to filter all files which have `test' or `Test' in their pathname.

%Figure~\ref{fig:LGmethod} shows a general overview of our approach, which consists of five steps: (1) We mine the git repository of each studied project to extract all commits made for each file.(2) We identify the log changes in the extracted files. (3) We track the changes made to each log across the commits. (4) We categorize the log changes in the commit and collect the process and change metrics for each log change in the commit. We use R~\cite{ihaka1996r}, to perform experiments and answer our preliminary analysis and case study.  


%Camel, CloudStack  and Liferay. Table~\ref{tba:overviewsystems} presents an overview of the projects.

%{ActiveMQ}



%Prior to performing our case study, we perform a preliminary analysis to evaluate how much logs change in the four studied projects.
%present the results of our preliminary analysis on the log changes in the four studied systems. 
%\subsection{Approach}
%\noindent \textsl{Approach\\}\\
%Using the tracked log changes, we look at how many times a single log can change in its lifetime. We use the tracked log data from Section~\ref{Methodology} to find the frequency of log changes in the studied projects. 



%Next, we look how many times can a single log change it its lifetime. 


%\begin{figure}[tb]
%	\centering
%%	\includegraphics[width=0.7\linewidth]{RQ1_Liferay_Camel_Logchangefreq}
%	\includegraphics[width=0\linewidth]{Percentageofchanges}
%	\caption{Percentage of log changes in the studied projects.}
%	\label{fig:RQ1_Liferay_Camel_Logchangefreq}
%\end{figure}

\begin{figure}[tb]
	
	\centering
%	\captionsetup{justification=centre}
	
%		\subfloat[Percentage of log changes]{\includegraphics[width=0.5\linewidth]
%			{Percentageofchanges}\label{fig:percentage} }

	\subfloat{\includegraphics[width=1\linewidth]
		{frequencyofLogchanges}\label{fig:Feq} }
	
	%	\includegraphics[width=0.7\linewidth]{RQ1_Liferay_Camel_Logchangefreq}
%	\includegraphics[width=0.9\linewidth]{frequencyofLogchanges}
	\caption{Distribution of log changes in the studied projects	} 
	\label{fig:frequencyofLogchanges}
\end{figure}



%To find how frequently logs change, we conduct a quantitative analysis on the studied systems. We use the tracked log data for each studied system as explained in Section~\ref{Methodology}. From each project, we select a random sample with 95\% confidence interval. We follow the same iterative process as in prior research~\cite{IanIcesm} to find how frequently logs change in our studied systems. 





%\begin{figure}[tb]
%	
%	\centering
%	%	\subfloat{\includegraphics[width=0.5\linewidth]{CA_numberofDevelopers}\label{fig:f1}}
%	%	\hfill
%	\subfloat{\includegraphics[width=1\linewidth]{ChangedvsChangesorNot}\label{fig:f2}}
%	\caption{Distribution of file ownership against developers introducing the log vs developers changing the log.}
%	\label{fig:ChangedvsChangesorNot}
%\end{figure}



%\begin{figure}[tb]
%	
%	\centering
%	%	\subfloat{\includegraphics[width=0.5\linewidth]{CA_numberofDevelopers}\label{fig:f1}}
%	%	\hfill
%	\subfloat{\includegraphics[width=1\linewidth]{NumberofDevelopers}\label{fig:f2}}
%	\caption{Distribution of the number of developers responsible
%		for changing a log.}
%	\label{fig:NumberofDevelopers}
%\end{figure}





\subsection{Results}

\begin{table}[tb]
	\centering
	\caption{Summary of the number of commits before a new log added is changed in the studied projects}
	
	\begin{tabular}{lrrrrrr}
			\cline{1-7}
		Project    & Min & 1st Qu & Median & Mean & 3rd Qu & Max \\
		\cline{1-7}
		ActiveMQ   & 1   & 2      & 7      & 9    & 14     & 37  \\
		Camel      & 1   & 1      & 2      & 4    & 5      & 117 \\
		Cloudstack & 1   & 1      & 3      & 17   & 14     & 390 \\
		Liferay    & 1   & 1      & 1      & 7    & 1      & 130\\	\cline{1-7}
		
	\end{tabular}
	\label{tba:summaryofnewLogchange}
\end{table}


%\subsubsection*{C.1. Change frequency}
\hypobox {Developers change 35\%-50\% of the logs across our studied projects. }
Figure~\ref{fig:frequencyofLogchanges} shows the percentage values for the number of times a log is changed in each of the studied projects. This shows that logs change extensively throughout the lifetime of an project which can affect the log processing tools.

\begin{table}[t]
	\centering
	\caption{Summary of total code churn in the commits where a new log is changed}
	
	\begin{tabular}{lrrrrrr}
		\cline{1-7}
		Project    & Min & 1st Qu & Median & Mean & 3rd Qu & Max \\
		\cline{1-7}
		ActiveMQ   & 2  & 25      & 47     & 141    & 163     & 493  \\
		Camel      & 2   & 13      & 32      & 98    & 133      & 456 \\
		Cloudstack & 2   & 66      & 234      & 410   & 574     & 4121 \\
		Liferay    & 2   & 6      & 14     & 28    & 27     & 278\\	\cline{1-7}
		
	\end{tabular}
	\label{tba:summaryofnewLogCodechange}
\end{table}




\textbf{75\% of the new logs which change, are changed within 15 commits since their addition.} From Table~\ref{tba:summaryofnewLogchange}, we find that majority, i.e., 75\% are changed within 15 commits since addition.  We also find that the median code churn during these log changes is less than 50 lines of code in three of the studied projects as seen in Table~\ref{tba:summaryofnewLogCodechange}. This suggests that the log changes are more likely to be changed due to rewording changes rather than major changes to the added feature. This means that new logs which are introduced prior to 15 commits in the studied projects, are less likely to break the log processing tools which might depend on them. 


%\textbf{25\% of the new logs which change, are changed after 15 commits since their addition.} From Table~\ref{tba:summaryofnewLogchange}, we find 25\% of the new logs added are changed after 15 commits since their addition. We also find that the code churn during these log changes is more than 150 lines of code in three of the studied projects as seen in Table~\ref{tba:summaryofnewLogCodechange}. This suggests that these log changes are more likely to be changes to the feature rather than rewording changes, and are more likely to affect the log processing tools. 

%This also means that the remaining 25\% of log changes might break the log processing tools which might rely on them, as they are changed much later. 



%rewording or after-thoughts rather than change of feature.

% This short time between addition and log change suggests that log change are more likely to be rewording or after-thoughts rather than change of feature. This is seen in Table~\ref{tba:summaryofnewLogCodechange} where in three of the projects, the code churn is less than 50 lines of code for 50\% of the new logs which are changed before 15 commits since addition. 



%We find that logs are added throughout the lifetime of an project and these logs are changed within 

%, the logs added near the end of our study time-frame can change but will not be considered in our analysis. We find that a new log is changed between 37 to 390 commits within our projects as seen in Table~\ref{tba:summaryofnewLogchange}. To eliminate such logs, we find the maximum number of commits before a newly added log is changed within the studied projects and exclude all new logs added before that many commits from our analysis. 


%From Table~\ref{tba:summaryofnewLogchange}, we see that the maximum value varies greatly within the studied projects and we exclude all newly added logs into the project before the 
 
%as shown in Figure~\ref{fig:RQ1_Liferay_Camel_Logchangefreq}. 



%Based on frequency of changes, we categorize logs into 3 categories namely: a) Frequently Changed, b) Changed and c) Never Changed as shown in Table~\ref{tba:logchangeDistribution}. If a log is changed more than four times it is categorized as `Frequently Changed'. If it is changed 1 to 3 times it is categorized under `Changed' and if it did not change it is categorized under `Never Changed'. We select four as the threshold as we observe that majority of logs only have 1 to 2 changes as seen in Figure~\ref{fig:RQ1_Liferay_Camel_Logchangefreq}. We see that the majority of logs never change in Liferay and the majority of logs in ActiveMQ and CloudStack are changed atleast once. This may be because Liferay has fewer logs per source code file (i.e., lower log density) when compared to ActiveMQ and CloudStack as seen in Table~\ref{tba:overviewsystems}. 


%\subsubsection*{C.2. Developer impact}
%After identifying the frequency of changes within the studied projects, we find the number of the developers responsible for the log changes and also if they own the file which contains the log. We use the developer name available from the `git log' to count the number of developers who change a log. To decide whether a developer owns a file we calculate the ratio of number of lines written by him to the total lines of code using the `blame' command available in Git. Since \emph{blame} only shows the changes made to a file in the last commit, to calculate the contribution of a developer to a file we recursively look at changes made in previous commits by that developer. We use the \emph{blame} to calculate the contribution at each commit and take the mean contribution across all commits to find his ownership of a file.

%we obtain all the the commits made to a file and calculate th



%\hypobox{Logs which change are introduced by developers who have little ownership over the file.} 
%Figure~\ref{fig:ChangedvsUnchangedlogs} shows that in Camel, Cloudstack and Liferay, the logs which change are more likely to be introduced by developers who have less ownership on the files, than logs which are never changed. This suggests that logs can be introduced by non-owners of a file, which leads to logs being changed later. 

%We see that logs are changed by developers who have lesser ownership over the 
%file than the developers who introduce the log.

%We see thats logs are also changed by developers who have lesser ownership than the ones introducing them. Figure~\ref{fig:ChangedvsChangesorNot} shows that in all the studied projects the logs are more likely to be changed by developers who have lesser ownership on the file than the developers who introduce the log. These results suggest that logs are readily changed by developers who access the file but do not have strong ownership characteristics.   



%These results suggest that logs are readily changed by developers who access the file but do not have strong ownership characteristics.

%Figure~\ref{fig:ChangedvsChangesorNot} shows that in all the studied


 

% which change are introduced by developers who have less ownership on files than the developers who introduce the log.
 
%  We also find that in one of the studied projects the majority of logs are changed by two or more developers as seen in Figure~\ref{fig:NumberofDevelopers}. 

% we see that in two of the studied systems, a single developer is responsible for majority of the log changes. 

%This suggests that logs can be 
% do not have strong ownership characteristics and can be changed by developers than one introducing the logs.

%\hypobox {45\%-55\% of the logs are changed atleast once in the studied projects. We find that over 51\% of the changes are made to the static content, variable content and log level. We also find that logs are changed by developers who have little ownership of the file and in two of the projects we find the majority of logs are changed by two or more developers.}

%When logs are changed, they can be changed in five possible ways namely:
%\begin{enumerate}
%	
%	\item { \textbf{Log relocation:} } The log is kept intact but moved to a different location in the file because of context changes (code around the log is changed).
%	
%	\item \textbf{Text change:} The text (i.e., static content) of log is changed. 
%	
%	\item\textbf{Variable change:} One or more variables in the log are changed (added, deleted or modified).
%	
%	\item \textbf{Change of log level:} The verbosity level of a log is changed.
%	
%	\item  \textbf{Text and variable change:} Both text and variables in the logs are changed. This is generally done when developers provide more context information, i.e, text and add/modify the relevant variables in a log.
%	
%\end{enumerate}
%
%%for several reasons. To understand the different types of log changes we perform a manual analysis on the changed logs. We select a random sample from each project such that the sample achieves 95\% confidence interval. After identifying the different types of log changes we automate the process of identification using our scripts. Figure~\ref{fig:Flowchart2} highlights the process of categorizing the log changes. For example,consider the logs shown below. 


%We see that there can be only four possible ways in which developers change logs namely:
%
%To automate the process of categorizing log changes into these categories, we first remove the logging method (i.e, LOG) and the log level (i.e, info) from the logs. We then compute the \textsl{Levenshtein ratio} between each term within the parentheses. In the example below we find that `+ Integer.toString(listenPort)' has \textsl{Levenshtein ratio} of 1, implying they are identical and the \textsl{Levenshtein ratio} between `starting HBase HsHA Thrift server on' and `starting HBase' is 0.56. This suggests there is some similarity between the two strings and the variable is constant which implies its a text change. (Figure~\ref{fig:Flowchart2} highlights the process of categorizing the log changes.
%
%\hypobox {+ LOG.info(``starting HBase HsHA Thrift server on " + Integer.toString(listenPort)); }
%
%\hypobox {- LOG.info(``starting HBase " + implType.simpleClassName() +`` server on " + Integer.toString(listenPort)); }



% ownership of file using the `blame' command available in git i.e.., if two developers are responsible for a file, but one has written 100 of the 150 lines of code, we calculate his 
%\begin{enumerate}
%	
%	\item { \textbf{Log relocation:} } The log is kept intact but moved to a different location in the file because of context changes (code around the log is changed).
%	
%	\item \textbf{Text change:} The text (i.e., static content) of log is changed. 
%	
%	\item\textbf{Variable change:} One or more variables in the log are changed (added, deleted or modified).
%	
%	\item \textbf{Change of log level:} The verbosity level of a log is changed.
%	
%	\item  \textbf{Text and variable change:} Both text and variables in the logs are changed. This is generally done when developers provide more context information, i.e, text and add/modify the relevant variables in a log.
%	
%\end{enumerate}
% 

%\noindent \textbf{Results}


%{\suhas{ I have question Can i tell here that in RQ2 we find log density to be imporatnt factor in stability of logs ?? }}

% are project layer software which  rely less on logs as they are middle-ware/project software, whereas ActiveMQ and CloudStack are service software.




%From manually analyzing the changed logs, we identify five types of log changes (i.e., changes to verbosity levels, log context, logged variables, both context and variable and relocation of log).  Table~\ref{tba:logtype} shows their distributions. When there is overlapping of the different types of log changes, we categorize them as newly added log and track changes made to it.

%	We find that about 3-11 \% of logs are changed frequently. This suggests that log processing tools which run on these systems need constant maintenance from developers




\section{Building a log change classifier}
%%In this section, we present our study results by answering our research questions. For each question, we discuss the motivation behind it, the approach to answering it and finally the results obtained. 
%\\
%
%\noindent\textbf{RQ1:} \textbf{How much do logs change over time and why do the changes occur?}
%\\

%\noindent\textbf{Motivation}

%Research has shown that logs evolve along with the code~\cite{IanContextinformation}. When logs are changed, the log  processing tools which are dependent on them also have to get updated. This results in costly maintenance effort. To understand the cost, we have to understand how frequent changes to logs are. Hence, we explore the frequency of changes to logs in our studied systems and why they are changed.\\

%\noindent \textbf{Approach}


%\begin{table*}
%	\centering
%	\caption{Distribution of log changes in different projects}
%	\label{tba:logchangeDistribution}
%	\begin{tabular}{l|>{\centering}p{.1\columnwidth}>{\centering}p{.01\columnwidth} 
%			p{0.01\columnwidth} }
%		\cline{1-4}  	\multicolumn{1}{|c}{Projects}    & \multicolumn{1}{|c}{Never Changed (\%) }  &  \multicolumn{1}{|c}{Changed (\%) }	   &  \multicolumn{1}{|c}{Frequently Changed (\%) }\\ \cline{1-4}   
%		
%		Life Ray      & 78.67     & 19.66 & 1.66           \\
%		
%		Camel      & 55.43    & 37.32 & 7.25            \\
%		ActiveMq   & 34.78     & 62.02 & 3.20           \\
%		CloudStack & 19.68     & 68.61 & 11.71          \\ \cline{1-4}
%	\end{tabular}
%\end{table*}

%\begin{table*}
%	\centering
%	\caption{Distribution of log changes in different projects}
%	\label{tba:logtype}
%	\begin{tabular}{l|llll}
%		\cline{1-5}  	\multicolumn{1}{|c}{Projects}    & \multicolumn{1}{|c}{ ActiveMQ }  &  \multicolumn{1}{|c}{ Camel}	   &  \multicolumn{1}{|c}{ Cloudstack }  & 
%		\multicolumn{1}{|c|}{ Liferay } \\ \cline{1-5}   
%		
%		Log relocation (\%)       & 20.05     & 20.80 &  53.32  & 49.42         \\
%		
%		Log text change (\%)      & 26.59    & 26.37 & 15.54    & 32.70       \\
%		Log variable change (\%)   & 53.18     & 52.75 & 31.08 &  16.75     \\
%		Change of log level (\%) & 0.18   & 0.08 & 0.04  &  1.13       \\ 
%%		Text and variable change (\%) & 2.33     & 6.39 & 22.0   &  30.59    \\ 
%\cline{1-5}
%	\end{tabular}
%\end{table*}


%In this paper we aim to get a better understanding of the
%unstable logs in an application so that we can improve the maintenance of log processing tools. 
In this paper we study the changes that are made to logs across multiple releases in open source projects. The goal of our study is to present a model for deciding whether a logging statement is likely to change in the future. This model can assist developers of log processing tools to decide on which logging statements they want their tool to depend. %In order to understand and build a random forest classifier to predict log changes, it is necessary to identify the extent logs are changed within these applications.
First, we perform a preliminary analysis, in which we examine how often logging statements change, to motivate our work. In this section, we present our rationale for selecting the projects that we studied and present the results of our preliminary analysis of the four studied projects. 

\subsection{Studied Projects}
In this paper, we examine the logging statements in open source projects. We selected these projects based on the following three criteria:
\begin{itemize}
	\item \textbf{Log usage} - The selected projects must make extensive use of logs in their source code
	%	This helps to improve the performace of the random forest classifier and to identify the factors which effect log stability.
	\item \textbf{Project activity} - The projects must have a mature development history (i.e., more than 10,000 commits)
	\item \textbf{Technology used} - To simplify the implementation of our study, we opted to select only projects that are written in Java and are available on a Git repository
%	We pick applications written in Java as it is one of the most popular languages today~\cite{Javaprog}.
%	 to identify and track the log changes across multiple releases. 
\end{itemize}

To select projects matching these criteria, we first selected all Java projects from the list of Apache Foundation Git repositories\footnote{\url{https://git.apache.org/}} that have more than 10,000 commits. Next, we counted the number of logging statements in a repository using the \code{grep} command in Listing~\ref{lst:grep}.

\vspace{-5mm}
\begin{Code}
\begin{lstlisting}[caption={Counting logging statements}, label={lst:grep}]
grep -icR 
"\(log\|s\_logger\)\.\(warn\|info\|error\|debug\|warn\)(" . 
| grep "\.java"
\end{lstlisting}
\end{Code}
\vspace{-5mm}




%To find the log usage in an project we use the \emph{grep} command to search all lines of code within the \emph{.java} files. %Next, using \emph{git log} we find the total number of commits in the code repositories and select projects which have more than 10,000 commits. 

We selected the four projects (ActiveMQ, Camel, Cloudstack and Liferay) with the highest number of logging statements for further analysis. ActiveMQ\footnote{\url{http://activemq.apache.org/}} is an open source message broker and integration patterns server. Camel\footnote{\url{http://camel.apache.org/}} is an open source integration platform based on enterprise integration patterns. CloudStack\footnote{\url{https://cloudstack.apache.org/}} is an open source project designed to deploy and manage large networks of virtual machines. Liferay\footnote{\url{http://www.liferay.com/}} is an open source platform for building websites and web portals. Table~\ref{tba:overviewsystems} presents an overview of the studied projects. %We pick the releases after incubation for each project, as during incubation the projects might not be used by log processing tools.  


\begin{table}[tb]
	\centering \protect\protect\caption{An overview of the studied projects}
	\smaller
	
	\label{tba:overviewsystems} %
	\begin{tabular}{lrrrr}
		\toprule 
		Projects  & ActiveMQ  & Camel  & CloudStack  & Liferay \\
		\midrule
		\# of logging statements  & 5.1k  & 6.1k  & 9.6k  & 1.8k \\
		\# of commits &	 ?  & ?  & ?  & ? \\	
		\# of months in repository &	 ?  & ?  & ?  & ? \\
%		\# of releases  & 19  & 43  & 111  & 24 \\
		\midrule
		\# of added lines of code  & 261k  & 505k  & 1.09M  & 3.9M \\
		\# of deleted lines of code  & 114k  & 174k  & 750k  & 2.8M \\
		\# of added logging statements  & 4.5k  & 5.1k  & 24k  & 10.4k \\
		\# of deleted logging statements  & 2.3k & 2.4k  & 17k  & 8.1k \\
		\% of logging-related changes & 1.8 & 1.1 & 2.3 & 0.3 \\
		\bottomrule 
	\end{tabular}
\end{table}

\subsection{Data Extraction Approach}

The data extraction approach from the four studied applications consists of three steps: (1) We clone the Git repository of each studied project to extract the change history made for each file (2) We identify the logs present in the file, and using the change history we identify the log changes made in each file (3) We track the log changes that are made to each log in a file across the commits. We use R~\cite{ihaka1996r}, to perform experiments and answer our preliminary analysis and empirical study. Figure~\ref{fig:LGmethod} shows a general overview of our approach and we discuss each step discussed above in further detail. 

\subsubsection*{B.1. Extracting the change history} 
In order to find the stability of logs, we have to identify all the Java files in our studied projects. To achieve this, we use the \emph{grep} command to search for all the \emph{*.java} files in the cloned repositories and we exclude the \emph{test} files. 

After collecting all the Java files from the four studied projects, we use their Git repositories to obtain all the changes that are made to the files within the time-frame shown in Table~\ref{tba:overviewsystems}. We use the \emph{follow} option to track a file even when it is renamed or relocated. We exclude log changes that are made in non-merged branches as they are unlikely to affect log processing tools. We use the \emph{--no-merges} option to flatten the changes to a file and exclude the final merging commit. Using this approach, we obtain a complete history of each Java file in the latest version of the master branch.


\subsubsection*{B.2. Identifying log changes}
From the extracted change history of each {Java} file, we identify all the log changes made in the commits. To identify the log statements in the source code, we manually sample some commits from each studied project and identify the logging library used to generate the logs. We find that the studied projects use \textsl{Log4j}~\cite{EMSEIAN} and \textsl{Slf4j}\footnote{http://www.slf4j.org/} widely and \textsl{logback}\footnote{http://logback.qos.ch/} sparingly. Using this information, we identify the common method invocations that invoke the logging library. For example, in  ActiveMQ and Camel a logging library is invoked by method named \emph{LOG} as shown below.
\hypobox{ LOG.debug(``Exception detail", exception);}


As a project can have multiple logging libraries throughout its life-cycle, we use regular expressions to match all the common log invocation patterns (i.e., \emph{LOG,log,\_logger,LOGGER,Log}). We consider every invocation of a logging library followed by a logging level (\emph{info, trace, debug, error, warn}) a log.


\subsubsection*{B.3. Tracking log changes}
After identifying all the log changes that are made to a file across multiple commits, we track each log individually to find out whether it has changed in subsequent revisions. We first collect all the logs present in a file at the first commit, which form the initial set of logs for the file. Every change to a log in the subsequent commits appears as an added and deleted log in Git. To identify added, deleted and modified logs, we leverage the Levenshtein ratio~\cite{levenshteinratio}. We use Levenshtein ratio instead of string comparison, because Levenshtein ratio quantifies the difference between the strings compared within the range 0 to 1 (more similar the strings the ratio approaches 1). This is necessary to compare multiple logs which can be similar, which is not possible using string comparison.

%within the range 0 to 1 (more similar the strings the ratio approaches 1).

%To calculate Levenshtein ratio for a pair of added and deleted logs, we first remove the logging method (e.g., LOG) and compare the remaining text to increase the accuracy of categorization. 


%We set a minimum threshold of 0.6 for a pair of added and deleted logs to be considered modified. We set 0.6 because there is atleast 60\% similarity between the pair. We find that when the threshold is set lower there are more false positives and when set higher we miss log modifications. When an added log has levenshtein ratio higher than 0.6 with multiple deleted logs, we consider the pair with highest levenshtein ratio. 
To identify log modification we calculate the Levenshtein ratio between all the deleted log and added logs and select the pair which has the highest Levenshtein ratio. This is done recursively to find all the modifications within a commit. For example in the logs shown below, we find that the Levenshtein ratio between the added and deleted pair (a1) is 0.86 and (a2) 0.76. Hence, we consider (a1) as a log modification and compare (a2) with next deleted log. If there are no more deleted log pairs, (a2) is considered as addition of new log into the file. 
\hypobox{-        LOG.debug(``Call: " +method.getName()+ `` " + callTime);\\
	+      LOG.debug(``Call: " +method.getName()+`` took "+ callTime + ``ms"); \space  \space  \space  \space  \space  \space -- \textbf{(a1)}\\ 
	+      LOG.debug(``Call: " +method.setName()+`` took "+ callTime + ``ms");\space  \space  \space  \space  \space  \space -- \textbf{(a2)}} 

This way we track when a log is added into a file and the log is added to the initial set for tracking in future commits. From this, we track how many times a log is changed and how many commits are made between the changes. 

As we cannot identify changes to a log which is added near the end of the time frame, we have to exclude such logs from our analysis. We find that in the studied projects, a new log added changes within 390 commits of being added, as seen from Table~\ref{tba:summaryofnewLogchange}. We exclude all logs added into the project 390 commits before the last commit of our analysis. 


%, we find that this varies widely within the project between 37 to 390 commits. To eliminate such logs, we use the maximum number of commits before a newly added log is changed within the studied projects and exclude all new logs added before that many commits from our analysis.  



%\subsubsection{Match JIRA to Log Changes}

%\subsection{Metric analysis}
%After collecting all the log changes, we look at the different types of log changed which occur in our studied projects. We find that log relocation occurs more often than other types of log changes in all of the studied projects. We find that log relocation occurs between 20\%-50\% as seen in Table~\ref{tba:logtype}. As log relocations have no changes to the text or variables in logs, their effect on log processing tools is limited. Hence we exclude log relocation changes from our datasets and non-relocation changes for the random forest classifier. 




%To achieve this, we clone the \emph{master} branch of the git repository of each studied project locally. We use the `find' command to recursively find all the files which end with pattern `*.java'. To remove the \textsl{Java Test} files, we use the `grep' command to filter all files which have `test' or `Test' in their pathname.

%Figure~\ref{fig:LGmethod} shows a general overview of our approach, which consists of five steps: (1) We mine the git repository of each studied project to extract all commits made for each file.(2) We identify the log changes in the extracted files. (3) We track the changes made to each log across the commits. (4) We categorize the log changes in the commit and collect the process and change metrics for each log change in the commit. We use R~\cite{ihaka1996r}, to perform experiments and answer our preliminary analysis and case study.  


%Camel, CloudStack  and Liferay. Table~\ref{tba:overviewsystems} presents an overview of the projects.

%{ActiveMQ}



%Prior to performing our case study, we perform a preliminary analysis to evaluate how much logs change in the four studied projects.
%present the results of our preliminary analysis on the log changes in the four studied systems. 
%\subsection{Approach}
%\noindent \textsl{Approach\\}\\
%Using the tracked log changes, we look at how many times a single log can change in its lifetime. We use the tracked log data from Section~\ref{Methodology} to find the frequency of log changes in the studied projects. 



%Next, we look how many times can a single log change it its lifetime. 


%\begin{figure}[tb]
%	\centering
%%	\includegraphics[width=0.7\linewidth]{RQ1_Liferay_Camel_Logchangefreq}
%	\includegraphics[width=0\linewidth]{Percentageofchanges}
%	\caption{Percentage of log changes in the studied projects.}
%	\label{fig:RQ1_Liferay_Camel_Logchangefreq}
%\end{figure}

\begin{figure}[tb]
	
	\centering
%	\captionsetup{justification=centre}
	
%		\subfloat[Percentage of log changes]{\includegraphics[width=0.5\linewidth]
%			{Percentageofchanges}\label{fig:percentage} }

	\subfloat{\includegraphics[width=1\linewidth]
		{frequencyofLogchanges}\label{fig:Feq} }
	
	%	\includegraphics[width=0.7\linewidth]{RQ1_Liferay_Camel_Logchangefreq}
%	\includegraphics[width=0.9\linewidth]{frequencyofLogchanges}
	\caption{Distribution of log changes in the studied projects	} 
	\label{fig:frequencyofLogchanges}
\end{figure}



%To find how frequently logs change, we conduct a quantitative analysis on the studied systems. We use the tracked log data for each studied system as explained in Section~\ref{Methodology}. From each project, we select a random sample with 95\% confidence interval. We follow the same iterative process as in prior research~\cite{IanIcesm} to find how frequently logs change in our studied systems. 





%\begin{figure}[tb]
%	
%	\centering
%	%	\subfloat{\includegraphics[width=0.5\linewidth]{CA_numberofDevelopers}\label{fig:f1}}
%	%	\hfill
%	\subfloat{\includegraphics[width=1\linewidth]{ChangedvsChangesorNot}\label{fig:f2}}
%	\caption{Distribution of file ownership against developers introducing the log vs developers changing the log.}
%	\label{fig:ChangedvsChangesorNot}
%\end{figure}



%\begin{figure}[tb]
%	
%	\centering
%	%	\subfloat{\includegraphics[width=0.5\linewidth]{CA_numberofDevelopers}\label{fig:f1}}
%	%	\hfill
%	\subfloat{\includegraphics[width=1\linewidth]{NumberofDevelopers}\label{fig:f2}}
%	\caption{Distribution of the number of developers responsible
%		for changing a log.}
%	\label{fig:NumberofDevelopers}
%\end{figure}





\subsection{Results}

\begin{table}[tb]
	\centering
	\caption{Summary of the number of commits before a new log added is changed in the studied projects}
	
	\begin{tabular}{lrrrrrr}
			\cline{1-7}
		Project    & Min & 1st Qu & Median & Mean & 3rd Qu & Max \\
		\cline{1-7}
		ActiveMQ   & 1   & 2      & 7      & 9    & 14     & 37  \\
		Camel      & 1   & 1      & 2      & 4    & 5      & 117 \\
		Cloudstack & 1   & 1      & 3      & 17   & 14     & 390 \\
		Liferay    & 1   & 1      & 1      & 7    & 1      & 130\\	\cline{1-7}
		
	\end{tabular}
	\label{tba:summaryofnewLogchange}
\end{table}


%\subsubsection*{C.1. Change frequency}
\hypobox {Developers change 35\%-50\% of the logs across our studied projects. }
Figure~\ref{fig:frequencyofLogchanges} shows the percentage values for the number of times a log is changed in each of the studied projects. This shows that logs change extensively throughout the lifetime of an project which can affect the log processing tools.

\begin{table}[t]
	\centering
	\caption{Summary of total code churn in the commits where a new log is changed}
	
	\begin{tabular}{lrrrrrr}
		\cline{1-7}
		Project    & Min & 1st Qu & Median & Mean & 3rd Qu & Max \\
		\cline{1-7}
		ActiveMQ   & 2  & 25      & 47     & 141    & 163     & 493  \\
		Camel      & 2   & 13      & 32      & 98    & 133      & 456 \\
		Cloudstack & 2   & 66      & 234      & 410   & 574     & 4121 \\
		Liferay    & 2   & 6      & 14     & 28    & 27     & 278\\	\cline{1-7}
		
	\end{tabular}
	\label{tba:summaryofnewLogCodechange}
\end{table}




\textbf{75\% of the new logs which change, are changed within 15 commits since their addition.} From Table~\ref{tba:summaryofnewLogchange}, we find that majority, i.e., 75\% are changed within 15 commits since addition.  We also find that the median code churn during these log changes is less than 50 lines of code in three of the studied projects as seen in Table~\ref{tba:summaryofnewLogCodechange}. This suggests that the log changes are more likely to be changed due to rewording changes rather than major changes to the added feature. This means that new logs which are introduced prior to 15 commits in the studied projects, are less likely to break the log processing tools which might depend on them. 


%\textbf{25\% of the new logs which change, are changed after 15 commits since their addition.} From Table~\ref{tba:summaryofnewLogchange}, we find 25\% of the new logs added are changed after 15 commits since their addition. We also find that the code churn during these log changes is more than 150 lines of code in three of the studied projects as seen in Table~\ref{tba:summaryofnewLogCodechange}. This suggests that these log changes are more likely to be changes to the feature rather than rewording changes, and are more likely to affect the log processing tools. 

%This also means that the remaining 25\% of log changes might break the log processing tools which might rely on them, as they are changed much later. 



%rewording or after-thoughts rather than change of feature.

% This short time between addition and log change suggests that log change are more likely to be rewording or after-thoughts rather than change of feature. This is seen in Table~\ref{tba:summaryofnewLogCodechange} where in three of the projects, the code churn is less than 50 lines of code for 50\% of the new logs which are changed before 15 commits since addition. 



%We find that logs are added throughout the lifetime of an project and these logs are changed within 

%, the logs added near the end of our study time-frame can change but will not be considered in our analysis. We find that a new log is changed between 37 to 390 commits within our projects as seen in Table~\ref{tba:summaryofnewLogchange}. To eliminate such logs, we find the maximum number of commits before a newly added log is changed within the studied projects and exclude all new logs added before that many commits from our analysis. 


%From Table~\ref{tba:summaryofnewLogchange}, we see that the maximum value varies greatly within the studied projects and we exclude all newly added logs into the project before the 
 
%as shown in Figure~\ref{fig:RQ1_Liferay_Camel_Logchangefreq}. 



%Based on frequency of changes, we categorize logs into 3 categories namely: a) Frequently Changed, b) Changed and c) Never Changed as shown in Table~\ref{tba:logchangeDistribution}. If a log is changed more than four times it is categorized as `Frequently Changed'. If it is changed 1 to 3 times it is categorized under `Changed' and if it did not change it is categorized under `Never Changed'. We select four as the threshold as we observe that majority of logs only have 1 to 2 changes as seen in Figure~\ref{fig:RQ1_Liferay_Camel_Logchangefreq}. We see that the majority of logs never change in Liferay and the majority of logs in ActiveMQ and CloudStack are changed atleast once. This may be because Liferay has fewer logs per source code file (i.e., lower log density) when compared to ActiveMQ and CloudStack as seen in Table~\ref{tba:overviewsystems}. 


%\subsubsection*{C.2. Developer impact}
%After identifying the frequency of changes within the studied projects, we find the number of the developers responsible for the log changes and also if they own the file which contains the log. We use the developer name available from the `git log' to count the number of developers who change a log. To decide whether a developer owns a file we calculate the ratio of number of lines written by him to the total lines of code using the `blame' command available in Git. Since \emph{blame} only shows the changes made to a file in the last commit, to calculate the contribution of a developer to a file we recursively look at changes made in previous commits by that developer. We use the \emph{blame} to calculate the contribution at each commit and take the mean contribution across all commits to find his ownership of a file.

%we obtain all the the commits made to a file and calculate th



%\hypobox{Logs which change are introduced by developers who have little ownership over the file.} 
%Figure~\ref{fig:ChangedvsUnchangedlogs} shows that in Camel, Cloudstack and Liferay, the logs which change are more likely to be introduced by developers who have less ownership on the files, than logs which are never changed. This suggests that logs can be introduced by non-owners of a file, which leads to logs being changed later. 

%We see that logs are changed by developers who have lesser ownership over the 
%file than the developers who introduce the log.

%We see thats logs are also changed by developers who have lesser ownership than the ones introducing them. Figure~\ref{fig:ChangedvsChangesorNot} shows that in all the studied projects the logs are more likely to be changed by developers who have lesser ownership on the file than the developers who introduce the log. These results suggest that logs are readily changed by developers who access the file but do not have strong ownership characteristics.   



%These results suggest that logs are readily changed by developers who access the file but do not have strong ownership characteristics.

%Figure~\ref{fig:ChangedvsChangesorNot} shows that in all the studied


 

% which change are introduced by developers who have less ownership on files than the developers who introduce the log.
 
%  We also find that in one of the studied projects the majority of logs are changed by two or more developers as seen in Figure~\ref{fig:NumberofDevelopers}. 

% we see that in two of the studied systems, a single developer is responsible for majority of the log changes. 

%This suggests that logs can be 
% do not have strong ownership characteristics and can be changed by developers than one introducing the logs.

%\hypobox {45\%-55\% of the logs are changed atleast once in the studied projects. We find that over 51\% of the changes are made to the static content, variable content and log level. We also find that logs are changed by developers who have little ownership of the file and in two of the projects we find the majority of logs are changed by two or more developers.}

%When logs are changed, they can be changed in five possible ways namely:
%\begin{enumerate}
%	
%	\item { \textbf{Log relocation:} } The log is kept intact but moved to a different location in the file because of context changes (code around the log is changed).
%	
%	\item \textbf{Text change:} The text (i.e., static content) of log is changed. 
%	
%	\item\textbf{Variable change:} One or more variables in the log are changed (added, deleted or modified).
%	
%	\item \textbf{Change of log level:} The verbosity level of a log is changed.
%	
%	\item  \textbf{Text and variable change:} Both text and variables in the logs are changed. This is generally done when developers provide more context information, i.e, text and add/modify the relevant variables in a log.
%	
%\end{enumerate}
%
%%for several reasons. To understand the different types of log changes we perform a manual analysis on the changed logs. We select a random sample from each project such that the sample achieves 95\% confidence interval. After identifying the different types of log changes we automate the process of identification using our scripts. Figure~\ref{fig:Flowchart2} highlights the process of categorizing the log changes. For example,consider the logs shown below. 


%We see that there can be only four possible ways in which developers change logs namely:
%
%To automate the process of categorizing log changes into these categories, we first remove the logging method (i.e, LOG) and the log level (i.e, info) from the logs. We then compute the \textsl{Levenshtein ratio} between each term within the parentheses. In the example below we find that `+ Integer.toString(listenPort)' has \textsl{Levenshtein ratio} of 1, implying they are identical and the \textsl{Levenshtein ratio} between `starting HBase HsHA Thrift server on' and `starting HBase' is 0.56. This suggests there is some similarity between the two strings and the variable is constant which implies its a text change. (Figure~\ref{fig:Flowchart2} highlights the process of categorizing the log changes.
%
%\hypobox {+ LOG.info(``starting HBase HsHA Thrift server on " + Integer.toString(listenPort)); }
%
%\hypobox {- LOG.info(``starting HBase " + implType.simpleClassName() +`` server on " + Integer.toString(listenPort)); }



% ownership of file using the `blame' command available in git i.e.., if two developers are responsible for a file, but one has written 100 of the 150 lines of code, we calculate his 
%\begin{enumerate}
%	
%	\item { \textbf{Log relocation:} } The log is kept intact but moved to a different location in the file because of context changes (code around the log is changed).
%	
%	\item \textbf{Text change:} The text (i.e., static content) of log is changed. 
%	
%	\item\textbf{Variable change:} One or more variables in the log are changed (added, deleted or modified).
%	
%	\item \textbf{Change of log level:} The verbosity level of a log is changed.
%	
%	\item  \textbf{Text and variable change:} Both text and variables in the logs are changed. This is generally done when developers provide more context information, i.e, text and add/modify the relevant variables in a log.
%	
%\end{enumerate}
% 

%\noindent \textbf{Results}


%{\suhas{ I have question Can i tell here that in RQ2 we find log density to be imporatnt factor in stability of logs ?? }}

% are project layer software which  rely less on logs as they are middle-ware/project software, whereas ActiveMQ and CloudStack are service software.




%From manually analyzing the changed logs, we identify five types of log changes (i.e., changes to verbosity levels, log context, logged variables, both context and variable and relocation of log).  Table~\ref{tba:logtype} shows their distributions. When there is overlapping of the different types of log changes, we categorize them as newly added log and track changes made to it.

%	We find that about 3-11 \% of logs are changed frequently. This suggests that log processing tools which run on these systems need constant maintenance from developers


%\/section{RQ2}
\label{prediction}
%	\noindent \textbf{RQ2:}\textbf{ Can metrics from code, log and developer dimension help in explaining the stability of logs?}
%	\\\\
%\noindent \textbf{Motivation}

In our preliminary analysis, we find that 25-45\% of the logging statements are changed in our studied applications. These logging statement changes affect the log processing tools that depend on the logs that are generated by these statements, forcing developers spend more time on maintenance of their tools. By analyzing the metrics which can help to decide whether a logging statement will change, developers of log processing tools can reduce the effort spent on maintenance by letting their tool depend on logging statements that are likely to remain unchanged. In this section, we train a random forest classifier for deciding whether a logging statement will change in the future. We then evaluate the performance of our random forest classifier and use the classifier to understand which metrics increase the likelihood of a change to a logging statement.

%understanding which factors can increase the likelihood of a log change, developers of log processing tools can mitigate the effort spent on maintenance by analyzing these factors and avoiding log  

%by performing critical analysis. 


% In this section we construct a random forest model for explaining the likelihood of log changes. We use the random forest model to also identify the most important metrics which explain	whether a log will change in the future. These insights into log 
	
%so that these factors can be taken into account by developers of log processing tools. 

% classifier and ident	ify the important metrics.

% we build a random forest classifier. 


% Hence, there is a need to identify these non-stable logs to simplify the job of developers. It can also benefit log processing tool developers to develop more robust applications making them more stable.

%
%can be extracted from control versions systems by developersthem.  \\


\subsection{Approach}
%\ian{Your categories of metrics should match the 3 things that you mentioned in intro! See my change below, make sure you change your text every where when you mentioned your metrics. }
We use metrics that measure the context, the content and the developers of the logging statements to train the random forest classifier. Context metrics measure the file context at the time of adding the logging statement. Content metrics collect information about the logging statement and developer metrics collect information about the developer adding the logging statement. Table~\ref{tba:Taxonomy} defines each metric collected and the rationale behind our choice of each metric. We use the Git repository to extract the context , content and developer metrics for the studied applications. These metrics benefit log processing tool developers as well as they do not need domain knowledge about the application to understand these metrics.


% We use context metrics as they describe the conditions in which the logging statement was added into the application. We use log metrics as they provide information about the added logging statement. 

We build random forest classifier~\cite{Albert2008424} to predict whether a logging statement will change in our studied applications. A random forest is a collection of decision trees in which the results of all trees are combined to form a generalized predictor. In our classifier, the context, content and developer metrics are the explanatory variables and the dependent class variable is a boolean variable that represents whether the logging statement ever changed or not (i.e., '0' for not changed and '1' for changed).




%and log metrics because this data can be extracted from the source code repository easily by developers. It also 

%To find the stability of logs in our studied systems, we extract code and log churn metrics from the Git repository and developer metrics from JIRA. We use code churn metrics because prior work has linked logs to development knowledge and issue reports~\cite{IanIcesm}.




%To build prediction models it is necessary to track the changes to logs within the time frame of our study. To achieve this we built a tool which tracks the changes made to each log.  
%
%Since no prior work exists( to our best knowledge ) to predict the stability of logs, we use the code and developer related metrics  to explain the stability of logs. Table ( make table for metrics) explain the different metrics in each category and the rationale for using them. Figure ( make figure explaining process of when log metrics collected and log tracked) shows the process of when metrics are collected and tracking of log. 

%	
%\textbf{Random Forest:} We build random forest to help in predicting the log changes in the revisions of studied systems. A random forest is collection of largely uncorrelated decision trees where they trees combine their results to form a generalized predictor (ref paper). Random forests use bagging strategy (breiman 1996), where the decision trees are constructed using a bootstrap sample dataset. The trees are independent i.e, they do not reply on the earlier trees. In addition to this, random forests split each node using the best among a subset of predictors randomly chosen at that node (breiman 2001). This makes the random forests robust against over-fitting and are more accurate than other tree algorithms (Brieman 2001). 

%\noindent{\textbf{{Model construction}}}

\begin{figure*}
	\centering
	\includegraphics[width=\textwidth]{ModelCreationLogGeanology}
	\caption{Overview of random forest classifier construction (C), analysis (A) and flow of data in random forest generation}
	\label{fig:ModelCreationLogGeanology}
\end{figure*}  

\begin{table*}
	\centering \protect\caption{The investigated metrics in our classifier}	
	\smaller
	\label{tba:Taxonomy} %
	\begin{tabular}{c|l|l|>{\raggedright}p{1\columnwidth}}
%		\hline 
\toprule	Dimension  & Metrics  & Values  & Definition (d) -- Rationale (r)\tabularnewline
%		\hline 
		\hline 
		\multirow{12}{*}{Context Metrics}

				&
%		Is readded	 & Boolean (0 -1) &
%		d: Check if the log is readded into a file \tabularnewline
%		\cline{4-4} &  &  &
%		r: Identify the logs which get readded into a file after they are removed. \tabularnewline
%		\cline{2-4} & 		
%		
%		Deleted count & Numerical &
%		d: Number of commits from the time a log is introduced into the file till it is deleted (removed) from the file. \tabularnewline
%		\cline{2-4} &  &  &
%		r: Find out how long it takes before a log is removed from the file. \tabularnewline
%		\cline{2-4} 
	
%		&
%		 New File & Boolean (0 -1) & 
%		d: Check if the log is added in a new file (i.e., newly committed)\tabularnewline
%		\cline{2-4} &  &  & 
%		r: This helps to identify which logs where added later in subsequent commits.\tabularnewline
%		\cline{2-4} 
%		& 
		Total revision count & Numerical & d: Total number of commits made to the file before the logging statement is added. This value is 0 for logging statements added in the initial commit but not for logging statements added overtime. \tabularnewline
%		\cline{4-4} 
		&  &  & r: Logging statements present in a file which is often changed, have a higher likelihood of being changed~\cite{Characterizinglogs}. Hence, the more commits to a file, the higher the likelihood of a change to a logging statement. 
%		chances of logs being changed in a commit as well.
%		Logs present in a file which is changed heavily, have higher chance of being changed.  This helps to find out if the file is changed heavily which can result in log changes~\cite{Characterizinglogs}.
		 \tabularnewline 	\cmidrule{2-4}
		& Code churn in commit & Numerical & d: The code churn of the commit in which a logging statement is added. \tabularnewline
%		\cline{4-4} 
		&  &  & r: The likelihood of change of logging statements that are added during large code changes, such as feature addition, can be different from that of logging statements added during bug fixes which have less code changes.
%		more stable than logs added during bug fixes which have lesser code changes.  
		 		\tabularnewline		\cmidrule{2-4}
		& Variables declared & Numerical & d: The number of variables which are declared before the logging statement in that function.
		\tabularnewline 
%		\cline{4-4} 
		&  &  & r: When a large number of variables are declared, there is a higher chance that any of the variables will be added to or removed from a logging statement afterwards. \tabularnewline
	\cmidrule{2-4}
		
		
		& SLOC & Numerical & d: The number of lines of code in the file.\tabularnewline
%		\cline{4-4} 
		&  &  & r: Large files have more functionality and are more prone to changes~\cite{zhang2009investigation}
		and changes to logging statements~\cite{Characterizinglogs,EMSEIAN}.
		\tabularnewline

	 	 \cmidrule{2-4}	 & Log context & Categorical & d: The block in which a logging statement is added i.e., \emph{if, if-else,
	 		 	try-catch, exception, throw, new function}.\tabularnewline
	 		 %		\cline{4-4} 
	 		 &  &  & r: 
%	 		 Prior research finds that logs are mostly used in assertion checks, logical branching and return value checking~\cite{Fu1}. 
	 		 The stability of logging statements used in logical branching and assertion checks, i.e., if-else blocks, may be different from the logging statements in try-catch, exception blocks. \tabularnewline
	
	
		\cmidrule{1-4}
Developer Metrics		& File ownership  & Numerical  & d: Percentage of the file written by the developer who added the logging statement. \tabularnewline
		%		\cline{4-4} 
		&  &  & r: The owner of the file is more likely to add stable logging statements than developers who have not edited the file before.
		\tabularnewline
		\cmidrule{2-4}
		& Developer experience & Numerical & d: The number of commits the developer has made prior to this commit. \tabularnewline
		%		\cline{4-4} 
		&  &  & r: More experienced developers may add more stable logging statements than a new developer who has less knowledge of the code. 			 		
		\tabularnewline
%		Research has shown that experienced developers might take up more
%		complex issues~\cite{rahman2011ownership} and therefore may leverage
%		logs more~\cite{EMSEIAN}.

%		\cline{2-4} 
	
	 
%		\hline
\midrule
		\multirow{6}{*}{Content Metrics}
		 & 
		 
		 Log addition & Boolean &
		 d: Check if the logging statement is added to the file after creation or it was added when file was created. \tabularnewline
		 &  &  & 
		 r: Newly added logging statements may be more likely to be changed than logging statements that exist since the creation of the file. \tabularnewline
		 %			Logs added into a file after creation might be more likely to be changed than the logs added during file creation.
%		 \tabularnewline 
		 
%		 \cmidrule{2-4} 
%		 &
%	Is re-added	 & Boolean &
%		d: Check if the logging statement is re-added into a file.  \tabularnewline
%	\cline{4-4}
%	 &  &  &
%	r: Logging statements which are added, removed and re-added into a file suggest that developers are unsure of the purpose of the log making them very unstable and prone to changes.  \tabularnewline
%\cmidrule{2-4}
%	& 	
%		Log change type	 & Categorical &
%		d: Check the type of log change the log has undergone before, i.e., relocation, text-variable change, level change. \tabularnewline
%		\cline{4-4}
%		 &  &  &
%		r: Changes to log text, variable and verbosity level can make logs more unstable than relocation changes.    
%		\tabularnewline
%\cmidrule{2-4}
%		\tabularnewline\tabularnewline

\cmidrule{2-4}
		& Log variable count & Numerical & d: Number of logged variables.\tabularnewline
%		\cline{4-4} 
		&  &  & r: Over 62\% of logging statement changes add new variables~\cite{Characterizinglogs}.
		Hence, fewer variables in the initial logging statement might result in
		addition of new variables later. \tabularnewline
\cmidrule{2-4}		& Log density & Numerical & d: Ratio of the number of logging statements to the source code lines in the file.\tabularnewline
%		\cline{4-4} 
		&  &  & r: Files that are well logged (i.e., with higher log density) may not need additional logging statements and are less likely to be changed.
		\tabularnewline
\cmidrule{2-4}
		& Log level & Categorical & d: The level (verbosity) of the added logging statement, i.e., \emph{info,error, warn, debug, trace} and \emph{trace}.\tabularnewline
%		\cline{4-4} 
		&  &  & r: Research has shown that developers spend significant amount of time in adjusting the verbosity of logging statements~\cite{Characterizinglogs}. Hence, the verbosity level of a logging statement may affect its stability. \tabularnewline
\cmidrule{2-4}
		& Log text count & Numerical & d: Number of text phrases logged. We count all text present between a pair of quotes as one phrase.\tabularnewline
%		\cline{4-4} 
		&  &  & r: Over 45\% of logging statements have modifications to static context~\cite{Characterizinglogs}.
	Logging statements with fewer phrases might be subject to changes later to provide a better explanation.\tabularnewline
\cmidrule{2-4} &
		Log churn in commit	 & Numerical &
		d: The number of logging statements changed in the commit. \tabularnewline
%		\cline{4-4}
		 &  &  &
		r:  Logging statements can be added as part of a specific change or part of a larger change. 
		\tabularnewline
\bottomrule
	\end{tabular}\protect
\end{table*}





Figure~\ref{fig:ModelCreationLogGeanology} provides an overview of the construction steps (C1 to C2) for building a random forest classifier and steps (A1 and A3) for analyzing the results. We use the statistical tool R to model and to analyze our data using the \emph{RandomForest} package.

\subsection*{Step C1 - Removing Correlated and Redundant Metrics}
%\textbf{\textsl{(C1 - Correlation analysis) }}

Correlation analysis is necessary to remove the highly correlated metrics from our dataset~\cite{correlationBook}. Correlated metrics can severely impact the calculation of importance in the random forest classifier, as small changes to one correlated metric can affect the values of the other correlated metrics, causing large changes on dependent class variable. 

%Collinearity between metrics can affect the performance of a model because small changes in one metric can affect the values of other metrics causing large changes on the dependent class variable. 

We use Spearman rank correlation~\cite{spearmanbook} to find correlated metrics in our data. Spearman rank correlation assesses how well two metrics can be described by a monotonic function. We use Spearman rank correlation instead of Pearson~\cite{pearsonbook} because Spearman is resilient to data that is not normally distributed. We use the function \emph{varclus} in R to perform the correlation analysis.

Figure~\ref{fig:SpearmanActiveMQ} shows the hierarchically clustered Spearman $\rho $ values in the ActiveMQ. The solid horizontal lines indicate the correlation value of the two metrics that are connected by the vertical branches that descend from it.  We include one metric from the sub-hierarchies which have correlation $|\rho| > 0.7 $. The dotted blue line indicates our cutoff value ($ | \rho | $ = 0.7). We use cutoff value of ($ | \rho | $ = 0.7) as used by prior research~\cite{ShaneOLS} to remove the correlated metrics before building our classifier.

We find that \emph{total revision count} is highly correlated with \emph{code churn in commit}, \emph{log churn in commit} and \emph{log addition}, because a file with more commits has higher chance of having a large commit with log changes, than a file with less commits. We exclude \emph{total revision count}, \emph{log churn in commit} and \emph{log addition} and retain \emph{code churn in commit} as it is a simpler metric to compute.

Correlation analysis does not indicate redundant metrics, i.e, metrics that can be explained by other explanatory metrics. The redundant metrics can interfere with the one another and the relation between the explanatory and dependent metrics is distorted. We perform redundancy analysis to remove such metrics. We use the \textsl{redun} function that is provided in the \textsl{rms} package to perform the redundancy analysis. We find after removing the correlated metrics, there are no redundant metrics.


\begin{figure}
	\centering
		\setlength{\belowcaptionskip}{-15pt}
	\includegraphics[width=0.95\columnwidth]{Spearman}
 \vspace*{-1cm}	\caption{Hierarchical clustering of variables according to Spearman’s $\rho$ in ActiveMQ}
	\label{fig:SpearmanActiveMQ}
\end{figure}


\subsection*{Step C2 - Random Forest Generation}
%\textbf{\noindent\textsl{(C2 - Random forest generation)}}

After we eliminate the correlated metrics from our datasets, we construct the random forest classifier. Random forest is a black-box ensemble classifier, which operates by constructing a multitude of decision trees on the training set and uses this to classify the testing set. From a training set of $m$ logging statements a random sample of \emph{n} components is selected with replacement~\cite{ShaneOLS} and using the \emph{randomForest} function from the \emph{randomForest} package in R, a random forest classifier is generated. 

%Figure~\ref{fig:BootRF} explains the construction of the random forest classifier, where 

%From the sampled selection, many classification trees are grown without pruning. Each classification tree gives a vote, i.e., \emph{true} if log is changed and \emph{false} if not. The random forest chooses the class with most number of votes over all the trees in the forest. This strategy helps to make the random forests more robust against over-fitting~\cite{breiman1996bagging}.

%and the steps are explained below.
%\begin{figure}
%	\centering
%	\includegraphics[height=.5\columnwidth,width=1\columnwidth]{BootRandomForest}
%	\caption{Overview of random forest generation in C2}
%	\label{fig:BootRF}
%\end{figure}


%Given a dataset of \textsl{m} logs for training, $ D = \{ (X_{1}Y_{1}),...,(X_{m}Y_{m}) \} $ where $X_{i, i = 1...m}$, is a vector of descriptors (i.e., $X$ are the metrics which are left after correlation analysis ) and $Y_{i}$ is the flag which indicates whether a log is changed or not.

%\begin{enumerate}
%\item From the training set of \textsl{m} logs, a random sample of \emph{n} components is selected with replacement (i.e., bootstrap sample)~\cite{ShaneOLS}.
%
%We use the \emph{boot} function from the \emph{boot} package in R to generate bootstrap samples. The boot function generates a set of random indices, with replacement from the integers 1:\emph{n}.
%
%\item From the sampled selection, many classification trees are grown without pruning. Each classification tree gives a vote, i.e., \emph{true} if log is changed and \emph{false} if not. The random forest chooses the class with most number of votes over all the trees in the forest. This strategy helps to make the random forests more robust against over-fitting~\cite{breiman1996bagging}.
%
% We use the \emph{randomForest} function from the \emph{randomForest} package in R to generate the random forest model. 
%
%
%\item The above steps are repeated until \textsl{M} such models are grown.
%\item Predict new data by aggregating the prediction of the $M$ models generated.
%\end{enumerate}  

%\begin{table}[t]
%	\centering
%	
%		\caption{Confusion Matrix}
%		\label{tba:confusion}
%		\resizebox*{\columnwidth}{.17\columnwidth}{
%	\begin{tabular}{|ll|l|l|}
%		\cline{1-4}
%		&                 & \multicolumn{2}{c|}{Classified}            \\ \cline{3-4} 
%		&                 & Logging statement changed         & Logging stsatement not changed     \\ \hline
%		\multicolumn{1}{|c|}{\multirow{2}{*}{Actual}} & Logging statement changed      & True positive (TP)  & False negative (FN) \\ \cline{2-2} 
%		\multicolumn{1}{|c|}{}                        & Logging statement not changed & False positive (FP) & True negative (TN)  \\ \hline
%	\end{tabular}}
%
%\end{table}
%
%%\begin{table}[t]
%
%	\begin{tabular}{|>{\raggedright}p{0.1\columnwidth}>{\raggedright}p{0.2\columnwidth}|l>{\raggedright}p{0.2\columnwidth}|}
%		\hline 
%		\multicolumn{1}{|>{\raggedright}p{0.03\columnwidth}}{\multirow{2}{0.03\columnwidth}{}} &  & \multicolumn{2}{c|}{Predicted}\tabularnewline
%		&  & Log has changed & Log has not-changed\tabularnewline
%		\hline 
%		\multirow{2}{0.03\columnwidth}{Actual} & Log has changed  & TP (True Positive)  & FN (False Negative ) \tabularnewline
%		& Log has not-Changed  & FP (False Positive )  & TN (True Negative) \tabularnewline
%		\hline 
%	\end{tabular}\protect
%	
%	
%
%\end{table}
\subsection*{Step A1 - Model Validation}
%\textbf{\noindent \textsl{(C3 - Model evaluation)}}

After we build the random forest classifier, we evaluate the performance of our classifier using precision, recall, F-measure, AUC and Brier Score. These measures are functions of the confusion matrix and are explained below.

\textsl{Precision (P)} measures the correctness of our classifier in predicting which logging statement will change in the future. Precision is defined as the number of logging statements which were correctly classified as changed over all logging statements classified to have changed as explained in Equation~\ref{f:precision}.   \\
	 \begin{equation}
 P =  \dfrac{TP}{TP + FP } 
 \label{f:precision}
	 \end{equation}	
	 
\textsl{Recall (R)} measures the completeness of our classifier. A classifier is said to be complete if the classifier can correctly classify all the logging statements which will get changed in our dataset. Recall is defined as the number of logging statements which were correctly classified as changed over all logging statements which actually change as explained in Equation~\ref{f:recall}.
	 \begin{equation}
	 R =  \dfrac{TP}{TP + FN } 
	 \label{f:recall}
	 \end{equation}	
	 
\textsl{F-Measure} is the harmonic mean of precision and recall, combining the inversely related measure into a single descriptive statistic as shown in Equation~\ref{f:Fmeasure}~\cite{F-Measure}.
	 \begin{equation}
	 F =  \dfrac{2 \times P \times R}{P + R } 
	 \label{f:Fmeasure}
	 \end{equation}	

\textsl{Area Under Curve (AUC)} is used to measure the overall ability of the classifier to classify changed and unchanged logging statements. AUC measures the discrimination, that is the ability of our classifier to classify if a logging statement changes or not. The value of AUC ranges between 0.5 (worst) for random guessing and 1 (best) where 1 means that our classifier can correctly classify every logging statement as changed or unchanged. We calculate AUC using the \emph{roc.curve} function from the \emph{pROC} package in R.





%a measure of the accuracy of the predictions in our model~\cite{wilks2011statistical}. It explains how well the model performs compared to random guessing i.e.,  a perfect classifier  will have Brier score of 0 and perfect misfit classifier will have Brier score of 1 (predicts probability of log change when log not changed). This means the lower the Brier score value, the better our random forest classifier.
	 
\textsl{Brier score (BS)} is a measure of the accuracy of the classifications of our classifier~\cite{wilks2011statistical}. The Brier score explains how well the classifier performs compared to random guessing as explained in Equation~\ref{f:BS}, where $P_{n}$ is the probability of whether a logging statement will change and $O_{n}$ is the boolean that shows whether the statement actually changed. A perfect classifier will have a Brier score of 0, a perfect misfit classifier will have a Brier score of 1 (predicts probability of log change when log is not changed). When there is an equal distribution of logging statements being changed and un-changed, the Brier score reaches the value of 0.25 for random guessing,(i.e., 50\% chance log is changed and 50\% changes log is un-changed). 

%the lower the Brier score value, the better our random forest classifier performs.

% This means lower the Brier score value, the  accuracy of random forest classifier and Brier score reaches the value 0.25 for random guessing. 

% The lower the Brier score value, the better our random forest classifier and reaches 0.25
%As the Brier score measure the total difference between the event and the forecast probability of the event, 
 
%The most common formulation of Brier Score is shown in Equation~\ref{f:BS}, where $f_{t}$ is the probability that was predicted, $o_{t}$ is the actual outcome of the event at the instance $t$ (0 if log is not changed and 1 if it is changed), and $N$ is the number of forecasting instances. 

%If the predicted value is 70\% and the log is changed, the Brier Score is 0.09 (lower the Brier score, the more likely the event will occur). It reaches 0.25 

	 \begin{equation}
	 BS = \sum_{n=1}^{M} (P_{n} - O_{n} )^{2} 
	 \label{f:BS}
	 \end{equation}	

\subsubsection*{Optimism}
The performance measures described previously may overestimate the performance of the classifier due to overfitting. To account for the overfitting in our classifier, we use the \textsl{optimism} measure, as used by prior research~\cite{ShaneOLS}. The \textsl{optimism} of the performance measures are calculated as follows:
\begin{enumerate}
	\item From the original dataset with \textsl{m} records, we select a bootstrap sample with \textsl{n} records with replacement.
	\item Build random forest as described in (C2) using the bootstrap sample.
	\item Apply the classifier built from the bootstrap sample on both the bootstrap and original data sample, calculating precision, recall, F-measure and Brier score for both data samples.
	\item Calculate the \textsl{optimism} by subtracting the performance measures of the bootstrap sample from the original sample. 	
\end{enumerate}

	The above process is repeated 1,000 times and the average (mean) \textsl{optimism} is calculated. Finally, we calculate \textsl{optimism-reduced} performance measures for precision, recall, F-measure, AUC and Brier score by subtracting the averaged optimism of each measure, from their corresponding original measure.  The smaller the optimism values, the less the original classifier overfits the data.

\begin{figure*}[tb]
	
	\centering
	\subfloat{\includegraphics[width=1.5\columnwidth]{BoxplotResults1}}
	\subfloat{\includegraphics[width=0.5\columnwidth,height=0.565\columnwidth]{BoxplotResults2}}
%	\hfill
%	\centering
%	\subfloat{\includegraphics[width=0.69\columnwidth]{CRBox} }
%	\hfill
%	\centering
%	\subfloat{\includegraphics[width=0.69\columnwidth]{CloRBox}
%		\label{fig:f42}}
%	\hfill
%	\centering
%	\subfloat{\includegraphics[width=0.69\columnwidth]{LRBox}\label{fig:f43}}
%	\hfill
	%	\caption{Distribution of the number of developers responsible for changing a log}
	\caption{The optimism reduced performance measures of the four applications}
	\label{fig:optmisim}
	
\end{figure*}

%\textbf{\noindent \textsl{(C4 - Importance of each metric in relation to log stability)}}
\subsection*{Step A2 - Identifying Important Metrics}

To find the importance of each metric in a random forest classifier, we use a permutation test. In this test, the classifier built using the bootstrap data (i.e., two thirds of the original data) is applied to the test data (i.e., remaining one third of the original data). Then, the values of the $X_{i}$ $^{th}$ metric of which we want to find importance for, are randomly permuted in the test dataset and the precision of the classifier is recomputed. The decrease in precision as a result of this permutation is averaged over all trees, and is used as a measure of the importance of metric $X_{i}$th in the random forest.


We use the \emph{importance} function defined in \emph{RandomForest} package of R, to calculate the importance of each metric. We call the \emph{importance} function every time during the bootstrapping process to obtain 1,000 importance scores for each metric in our dataset. 	



As we obtain 1,000 data sets for each metric from the bootstrapping process, we use the \textbf{Scott-Knott Effest size clustering} (SK-ESD) to group the metric based on their effect size~\cite{tantithamthavorn2015tse}. Such an approach groups metrics based on their importance in predicting the likelihood of logging statement changes. The SK-ESD algorithm uses effect size calculated using \emph{cohen's} delta~\cite{cohenUsage1}, to merge any two statistically indistinguishable groups. We use the \emph{SK.ESD} function in the \emph{ScottKnottESD} package of R and set the effect size threshold parameter to negligible, (i.e., $<$ 0.2) to cluster the two metrics into the same groups. 
%\begin{table}

%	\centering


\subsection*{Step A3 - Plotting the Important Metrics}

To understand the effect of each metric in our random forest classifier, it is necessary to plot the predicted probabilities of a change to a logging statement against the metrics. By plotting the predicted probabilities of a change to a logging statement, we obtain a clearer picture of how the random forest classifier uses the important metrics to classify the data. 

Using the \emph{randomForest} package in R, we build a classifier as explained in C2, and we use the \emph{predict} function in \emph{R}, to calculate the probabilities of a change to a logging statement. We plot each predicted probability against the value of the metric, to understand how changes in the metric values affect the probability of a change to a logging statement.


% We use the \emph{randomForest} package in R to build the classifier



\begin{table*}[t]
	\protect\protect\caption{The most important metrics, divided into homogeneous groups by Scott-Knott Effect Size clustering}
	
	\centering
	
	\begin{tabular}{lll|lll}
		\hline 
		& \textbf{ ActiveMQ}  &  &  & \textbf{Camel}  & \tabularnewline
		\hline 
		\textbf{Rank}  & \textbf{Factors}  & \textbf{Importance}  & \textbf{Rank}  & \textbf{Factors}  & \textbf{Importance} \tabularnewline
		\hline 
		1  & Developer experience  & 0.246   & 1  & Developer experience  & 0.272 \tabularnewline
		2  & Ownership of file  & 0.175   & 2  & Ownership of file  & 0.151 \tabularnewline
		3& Log density  & 0.163   & 3 & Log level & 0.138  \tabularnewline
		4  & Log variable count  & 0.101  & 4  & SLOC  & 0.112  \tabularnewline
		5  & Log level & 0.063   & 5 & Log addition & 0.090 \tabularnewline
		6 & Variable declared  & 0.048   &  & Log density & 0.088  \tabularnewline
		7  & Log context & 0.069   & 6  & Log variable count & 0.063 \tabularnewline
		8  & Log text length  & 0.022  & 7 & Log context & 0.052 \tabularnewline
		  &  &   & 8 & Variable declared & 0.051  \tabularnewline
		\hline 
	\end{tabular}
	
	\begin{tabular}{lll|lll}
		\hline 
		& \textbf{CloudStack}  &  &  & \textbf{Liferay}  & \tabularnewline
		\hline 
		\hline 
		\textbf{Rank}  & \textbf{Factors}  & \textbf{Importance}  & \textbf{Rank}  & \textbf{Factors}  & \textbf{Importance} \tabularnewline
		\hline 
		1  & Log density & 0.224  & 1  & Log density  & 0.192  \tabularnewline
		2  & Ownership of file  & 0.215   &  & Developer experience  & 0.195  \tabularnewline
		3& SLOC & 0.192   & 2 & Ownership of file  & 0.190  \tabularnewline
		4  & Developer experience  & 0.182   &  & SLOC & 0.188 \tabularnewline
		5 &  Log text length  & 0.120   & 3  & Log variable count  & 0.162  \tabularnewline
		6 & Log variable count   & 0.115   & 4 & Log level  & 0.148  \tabularnewline
		7  & Log level  & 0.102   &5 & Log context  & 0.091  \tabularnewline
		8 & Variable declared & 0.092   & 6  & Variable declared & 0.080 \tabularnewline
		9 & Log context & 0.061   & 7  & Log text length & 0.071 \tabularnewline
		\hline 
	\end{tabular} \label{tba:Scott} 
	
	
	
\end{table*}



%\label{tba:optmisim}
%\end{table}


\subsection{Results}

%<<<<<<< HEAD
%\hypobox{The random forest classifier achieves a precision of 0.89-0.91, recall of 0.71-0.83 and outperforms random guessing for our studied applications} 
%Figure~\ref{fig:optmisim} shows the optimism-reduced values of \textsl{precision}, \emph{recall}, \emph{F-measure} and \emph{Brier score} for studied application. The classifier achieves an AUC of 0.94-0.95 and Brier scores between 0.04 and 0.07 across all studied applications. If the classifier achieves a Brier score of 0.07, it means our classifier can forecast with 73\% probability a logging statement will change. Brier score reaches 0.25 for random guessing (i.e., predicted value is 50\%).
%=======
\hypobox{The random forest classifier achieves 0.89-0.91 precision, 0.71-0.83 recall and outperforms random guessing for our studied applications.} 
Figure~\ref{fig:optmisim} shows the optimism-reduced values of \textsl{precision}, \emph{recall}, \emph{F-measure} \emph{AUC} and \emph{Brier score} for each studied application. The classifier achieves an AUC of 0.94-0.95.

The Brier scores for ActiveMQ, Camel, Cloudstack and Liferay are 0.061, 0.060, 0.051 and 0.042 respectively. Using Equation~\ref{f:BS},  we find that for intuitive prediction based on the portion of changed and un-changed logging statements, the Brier score values are 0.20, 0.12, 0.24, 0.17 in ActiveMQ, Camel, Cloudstack and Liferay respectively. We find that our approach outperforms such intuitive prediction with a much lower Brier score. 


% 0.04 and 0.07 across all studied applications. Using the equation~\ref{f:BS}, we find the Brier score values for 

%we can find that a Brier score of 0.07 means that our model can decide with 73\% probability whether a logging statement will change. The Brier score is 0.25 for random guessing (i.e., predicted value is 50\%).

%\ian{re do the last sentence. Feels like repeating your approach. In addition, in your data, a predicted value of random guessing is not 50\%, since your data is not half half balanced. I would re-calculate the bscore for random guessing here and explain it better.}


%>>>>>>> 0350b6ebb31e99b3546a9d56cc3e31cef9adda6b
% We find the recall of the random forest classier in Liferay is not as high as the other projects. This may be because Liferay has the lowest total number of log lines and close to 50\% of the log changes are log relocations as seen in Table~\ref{tba:logtype}. Because of the lower percentage of logs which are changed, the random forest classifier has fewer nodes (trees) and likelihood of false negatives is higher. 

%lowest percentage of logs which are changed at 20\%, compared to 45-70\% in the other projects. Because of the lower percentage of logs which are changed, the random forest classifier has fewer nodes (trees) and likelihood of false negatives is higher. 

%\hypobox{The random forest classifier outperforms random guessing.} 

\begin{table}[tb]
	\centering
	\caption{Contribution of top 3 developers}
	
	\begin{tabular}{lr>{\centering}p{.25\columnwidth}r>{\centering}p{.2\columnwidth}}		
		\cline{1-4} 	& Total logs & Changed logging  & Total \# of\\ &&statements&   contributors \\  \cline{1-4}
		ActiveMQ    & 956 (50.4\%)            & 301 (31.4\%)   & 41 \\
		Camel       & 3,060 (63.1\%)              & 1,460 (47.7\%) &   151	    \\
		Cloudstack  & 5,982 (35.7\%)              & 2,276 (38.0\%)  &  204  \\
		Liferay     &  3,382 (86.7\%)            & 609 (18.0\%)     &  351 \\\cline{1-4}
		\textbf{Average}& \textbf{3,345 (59\%)} & \textbf{1,161} (\textbf{33.75\%})& 747 \\\cline{1-4}	
	\end{tabular}
	\label{tba:topDevelopers}
\end{table}


\subsection*{{B2. Important Metrics for Logging Statement Stability}}

%From Table~\ref{tba:Scott} we see that `file ownership', `SLOC', `log density', and `developer experience' are common across all the projects in our studied systems. 

%The high importance of \emph{SLOC}, \emph{code churn in commit} and \emph{\# of comments} indicate that log stability 

%% LyX 2.1.2 created this file.  For more info, see http://www.lyx.org/.
%% Do not edit unless you really know what you are doing.
\begin{figure}[tb]
				\setlength{\belowcaptionskip}{-10pt}
	\centering
	\subfloat{\includegraphics[width=0.49\columnwidth]{QP_Amq_Dxp}}
	\hfill
	\centering
	\subfloat{\includegraphics[width=0.49\columnwidth]{QP_Camel_Dxp} }
	\hfill
	\centering
	\subfloat{\includegraphics[width=0.49\columnwidth]{QP_Clo_Dxp}
	}
	\hfill
	\centering
	\subfloat{\includegraphics[width=0.49\columnwidth]{QP_Lif_Dxp}}
	\hfill
	%	\caption{Distribution of the number of developers responsible for changing a log}
	\caption{Comparing the probability of a change to a logging statement against the experience of the developer who adds that logging statement}
	\label{fig:Dexp_BP}
	
\end{figure}


%\begin{figure}[tb]
%	\centering
%	\includegraphics[width=1\linewidth]{ChangedvsUnchangedlogs}
%	\caption{Comparing the file ownership of developers who add logs that are never changed vs logs which are changed.}
%		\label{fig:ChangedvsUnchangedlogs}
%\end{figure}






%\textbf{{{Developer experience} is an important metric to explain the likelihood of a change to a logging statement.}}
\hypobox{In three out of four studied applications, the top three developers were responsible for adding over 50\% of the logging statements. Up to 70\% of these logging statements never change.}
Table~\ref{tba:Scott} shows the metrics that are important for deciding whether a logging statement will change in the future. From Table~\ref{tba:Scott}, we see that developer experience is in the top four metrics for all studied applications to help explain the likelihood of a logging statement being changed in all the studied applications. 
Figure~\ref{fig:Dexp_BP} shows the probabilities of a logging statement being changed as developer experience increases. In all the studied applications, logging statements that are added by new developers have a lower probability of being changed, when compared to those added by more experienced developers.
We account this phenomenon to the fact that inexperienced developers add a very small fraction of the logs (12\% in Cloudstack and 1-3\% in the other applications).


We also observe that as developers get more experience the probability of a change to a logging statement decreases in ActiveMQ, Camel and Liferay. This downward trend may be explained by the fact that in ActiveMQ, Camel and Liferay, the top three developers are responsible for adding more than 50\% of the logging statements as seen in Table~\ref{tba:topDevelopers}. In addition, we find that up to 70\% of the logging statements added by these top developers never change. These findings suggest that developers of log processing tools should let their tools depend on logging statements added by developers with a high level of experience.

%This suggests that logs which change are added by less experienced developers in ActiveMQ, Camel and Liferay. This can be seen in Figure~\ref{fig:Dexp_BP}, where logs which change three times or more are changed by less experienced developers, when compared to unchanged logs. For example, when we inspect \emph{git diff} for the file \emph{SessionBatchTransactionSynchronization.java} in Camel, we find that all the logs introduced by a new developer are fixed by a more experienced developer. In this case, the experienced developer changes the log contexts, i.e., relocates the logs and adds additional context information to provide more meaning to the logs. 


%We also find that in ActiveMQ, Camel and Liferay, the top three developers are responsible for more than 50\% of the logs. Only 30\% of the logs added by these developers are changed as seen in Table~\ref{tba:topDevelopers}. This result recommend that new developers should get more experience about the application by actively making more commits to write more stable logs. 


%\begin{table}[tb]
%	\centering
%	\caption{Contribution of developers having less than 25\% ownership in a file}
%	
%	\begin{tabular}{lrrr}		
%		\cline{1-4} 	& Total logs & Changed logs &  \\  \cline{1-4}
%		ActiveMQ    & 1402 (73.9\%)            & 400 (28.5\%)   & 41 \\
%		Camel       & 3060 (63.1\%)              & 1460 (47.7\%) &   151	    \\
%		Cloudstack  & 5982 (35.7\%)              & 2276 (38.0\%)  &  204  \\
%		Liferay     &  3382 (86.7\%)            & 609 (18.0\%)     &  351 \\\cline{1-4}
%		\textbf{Average}& \textbf{3345 (59\%)} & \textbf{1161} (\textbf{33.75\%})& 747 \\\cline{1-4}
%		
%	\end{tabular}
%	\label{tba:bottomOwnership}
%\end{table}



%\textbf{{Ownership of the file is an important metric to explain the likelihood of a change to a logging statement.}}
\hypobox{Logging statements that are added into a file by developers who own more than 75\% of that file are unlikely to be changed in the future.}

\begin{figure}[tb]
			\setlength{\belowcaptionskip}{-10pt}
	\centering
	\subfloat{\includegraphics[width=0.49\columnwidth]{QP_Amq_Ow}}
	\hfill
	\centering
	\subfloat{\includegraphics[width=0.49\columnwidth]{QP_Camel_Ow} }
	\hfill
	\centering
	\subfloat{\includegraphics[width=0.49\columnwidth]{QP_Clo_Ow}
	}
	\hfill
	\centering
	\subfloat{\includegraphics[width=0.49\columnwidth]{QP_Lif_Ow}}
	\hfill
	%	\caption{Distribution of the number of developers responsible for changing a log}
	\caption{Comparing the probability of a change to a logging statement against ownership of the file in which the logging statement is added }
	\label{fig:Numberofchanges_BP}
%		\vspace{-0.3cm}
\end{figure}


 From Table~\ref{tba:Scott}, we see that ownership of the file is in the top two metrics to help explain the likelihood of a change to a logging statement in all the studied applications. From Figure~\ref{fig:Numberofchanges_BP}, we observe in all the applications that logging statements introduced by developers who own more than 75\% of the file are less likely to be changed. We also observe that developers who own less than 20\% of the file are responsible for 27\%-67\% of the changes to logging statements in the studied applications, which is seen as upward trend from 0 to 0.20 in Figure~\ref{fig:Numberofchanges_BP}. These results suggest that developers of log processing tools should be more cautious when using a logging statement written by a developer who has contributed less than 20\% of the file.

 

% and logs introduced by developers who own less than 25\% are more likely to be changed. 
%
%\begin{figure}[tb]
%	
%	\centering
%%	\subfloat{\includegraphics[width=0.49\columnwidth]{QP_Amq_SLOC}\label{fig:f4}}
%	\subfloat{\includegraphics[width=0.49\columnwidth]{QP_Amq_Ld}\label{fig:f4}}
%	\hfill
%	\centering
%%	\subfloat{\includegraphics[width=0.49\columnwidth]{QP_Camel_SLOC} }
%		\subfloat{\includegraphics[width=0.49\columnwidth]{QP_Camel_Ld}\label{fig:f4}}
%	\hfill
%	\centering
%%	\subfloat{\includegraphics[width=0.49\columnwidth]{QP_Clo_SLOC}}
%			\subfloat{\includegraphics[width=0.49\columnwidth]{QP_Clo_Ld}\label{fig:f4}}
%%
%	\hfill
%	\centering
%%	\subfloat{\includegraphics[width=0.49\columnwidth]{QP_Lif_SLOC}\label{fig:f4}}
%		\subfloat{\includegraphics[width=0.49\columnwidth]{QP_Lif_Ld}\label{fig:f4}}
%	\hfill
%	%	\caption{Distribution of the number of developers responsible for changing a log}
%	\caption{Comparing file ownership of developers who add unchanged logs against logs changing more than once}
%	\label{fig:LD}
%	
%\end{figure}
%


%  suggests that logs added by developers who have little ownership are more unstable and more likely to be changed. This is show by Figure~\ref{fig:Numberofchanges_BP}, where in Camel and Liferay, the unchanged logs are introduced by developers who have more ownership of the file. We also find that the most unstable logs i.e., (3 + changes ) in Camel, Cloudstack and Liferay are done by developers who have lesser ownership of the file, accounting for the negative correlation in these three applications. These results suggest that developers who are not owners of a file, should be more cautious	 when adding or changing logs in the file. For example when we inspect the \emph{git diff} for the file \emph{SSLContextServerParameters.java} in Camel, we find that the owner of the file fixes the changes made by a non-owner by increasing the logging level to reduce a flood of logs being generated. 


%This is seen in Figure~\ref{fig:ChangedvsUnchangedlogs}, where in Camel and Liferay, the logs which change are more likely to be added by developers who have less ownership on the files, than logs which are never changed. We also find that when logs change, the developers making the changes have less ownership than developers adding the log. From Figure~\ref{fig:Numberofchanges_BP} we see that in Camel, Cloudstack and Liferay, the logs which change more than three times are done by developers who have lesser ownership of the file, accounting for the negative correlation in these applications. These results suggest that developers who are not owners of a file, should be more cautions when adding or changing logs in the file. For example when we inspect the \emph{git diff} for the file \emph{SSLContextServerParameters.java} in Camel, we find that the owner of the file fixes the changes made by a non-owner by increasing the logging level to reduce a flood of logs being generated. 

%when logs are changed by developer who has less ownership, an experienced developer reverts fixes the same logs again. 


% where we find that in Camel, Cloudstack and Liferay unchanged logs are introduced by developer who have higher ownership than developers introducing changed logs. 

%After identifying the frequency of changes within the studied applications, we find the number of the developers responsible for the log changes and also if they own the file which contains the log. We use the developer name available from the `git log' to count the number of developers who change a log. To decide whether a developer owns a file we calculate the ratio of number of lines written by him to the total lines of code using the `blame' command available in Git. Since \emph{blame} only shows the changes made to a file in the last commit, to calculate the contribution of a developer to a file we recursively look at changes made in previous commits by that developer. We use the \emph{blame} to calculate the contribution at each commit and take the mean contribution across all commits to find his ownership of a file.



% which implies a) logs introduced by more experienced developers are more likely to change or b) experienced developer introduce more logs in the studied application.

%We find that the top 3 developers introduce 59\% of the total logs in the studied applications as seen in Table\ref{tba:topDevelopers}. We also find that close to 64\% of the logs which change are from these experienced developers. This results suggest that even though experienced developers introduce majority of logs, they is no increase in log stability. 



%where in three of the studied applications, the logs which change are introduced by developers who have lower ownership of a file. 



%which are changed as seen in Table~\ref{tba:topDevelopers}. This suggests that even though 

%Even though ActiveMQ and Liferay has positive correlation in Table~\ref{tba:Scott}, we find that these projects have strong code ownership and  two developer are responsible for over 50\% of the total commits within the projects. To remove this strong ownership, we exclude the log changes made by these top developers in ActiveMQ and Liferay and we find that developer experience has negative correlation in both ActiveMQ and Liferay. This suggests that logs which are introduced by more experienced developers are less likely to change in all of the studied applications. 


%
%\textbf{{{Log density} is an important metric to explain the likelihood of a change to a logging statement.}}
\hypobox{Logging statements in files with a low log density are more likely to change than logging statements in files with a high log density.}

 From Table~\ref{tba:Scott}, we observe that log density has the highest importance in Liferay and Cloudstack. We find that in these two applications, changed logging statements are in files that have a lower log density than the files containing unchanged logging statements. When we measure the median file sizes, we find that logging statements which change more are present in files with significantly higher SLOC (2$\times$-3$\times$ higher). This suggests that large files that are not well logged are more likely to have unstable logging statements, than well logged files.



%From Figure~\ref{fig:LD} we see that increase in log density increases the likelihood of log change in ActiveMQ and Liferay. In these systems, we find that the files with highest log densities have significantly lower SLOC (10x smaller) than the median, implying that too many logs can decrease code readability 

% and decreases the likelihood of log change in Cloudstack and Camel. When we measure the median file sizes for Cloudstack and Camel, we find that logs which change more are present in files with significantly higher SLOC (2x-3x higher). 



%We find that log density has negative correlation with log stability (i.e., increase in log density decreases probability of log change), in Camel, Cloudstack and Liferay as seen in Table~\ref{tba:Scott}. We find that in these applications, the logs which change are present in files with lower log density than unchanged logs. When we measure the median file sizes we find that, logs which change more are present in files with significantly higher SLOC (2x-3x higher). T 
 
%have significanlty 

%more than three times are present in files with low density i.e., not a well logged file but more SLOC than files with unchanged logs.

 
% the log density for logs changing 1-2 times is higher than unchanged logs, but w


%{{log variable count} has a positive correlation with log stability as shown in Table~\ref{tba:Scott}.}
%This implies that more variables in a log results in a higher likelihood that a log will be changed. This may be because there are inconsistencies between logs and the actual needed information intended as shown by prior research~\cite{Characterizinglogs} and developers have to update logs to resolve the inconsistencies. We find that the median for changed logs in all the subject systems is two variables for 


%\textbf{We find that \emph{SLOC}(source lines of code) is a strong predictor of log change across all projects.} \emph{SLOC} has a positive correlation suggesting that logs in larger files have higher a likelihood of getting changed. 

%We find that \emph{Variable declared} has positive correlation in three of the studied systems, which suggests that when developers add new variables in the commit there is a higher likelihood of log change as they may add or modify the log to output the new data.
%, logs around those comments are less likely to get changed. 


% is also a strong predictor of log change in all our models but the correlation is split within the projects. We find that in Cloudstack and Camel, developer experience has a negative correlation on log change, where as in ActiveMQ and Liferay developer experience has positive correlation. 
%The positive correlation may be because of strong code ownership seen ActiveMQ and liferay where two developer are responsible for over 50\% of the total commits within the projects. To remove this strong ownership, we exclude the log changes made by these top developers in ActiveMQ and Liferay and we find that developer experience has positive correlation in ActiveMQ. This suggests that logs which are introduced by more experienced developers are less likely to change in three of the studied systems. 

 

%We find that positive correlation might be due strong code ownership in ActiveMQ and Liferay where two developers are responsible for over 50\% of the total commits, within these projects. The negative effect in ActiveMQ and Cloudstack suggests that logs written by more experienced developers are less likely to be changed. 



% This suggests that either (1) more experienced developers have logged the file increasing the stability of the logs or (2) less experienced developers have logged the file making logs unstable. 
%
%In these systems we find that, more logs are introduced by developers with less experience than developers greater experience. 


%This may be because (1) logs that are added by less experience developers are less stable or (2) more experienced developers might not be as careful about logging as new developers.

 

%
% This implies that logs introduced by more experienced developers and are more likely to be changed.  This may be because the experienced developers may be more complacent than less experienced developers, who thoroughly log the code. 

%We find that other metrics from the developer dimension are not consistent among the studied systems. This may be because each project might have different philosophy of development, for example we find that \emph{resolution time} has negative effect in Liferay and Camel but has positive effect in ActiveMQ. This suggests that logs are more likely to change in ActiveMQ when the resolution time of issue increases, but less likely in Liferay and Camel. 


\hypobox {Developer experience, file ownership, SLOC, and log density are important metrics for deciding whether a logging statement will change in the future.}










\section{Related Work \cp{FIXME (discuss relation)}}
\label{related}

In this section, we present prior research that performs log analysis on large software systems and tools developed to assist in logging. 


\subsection{Log Analysis}

% The purpose of this section is to highlight the research that shows
%logs are used in debugging process. This is related to our work because
%(1) we show that logs are used in all the processes of a software
%life cycle, (2) how logs are useful in the improving the quality of
%the software. 
%\ian{some names are textsl, some are not. I think just keep the name as regular font and put et al. as italic, sometimes you miss the . in et al. and you should put dollar signs before and after the .}
Prior work leverages logs for testing and detecting anomalies in large scale systems. Shang\textsl{ et al$.$}~\cite{IanContextinformation} propose an approach to leverage logs in verifying the deployment of Big Data Analytics applications. Their approach analyzes logs in order to find differences between running in a small testing environment and a large field environment. Lou \textsl{et al}$.$~\cite{JGLouMining} propose an approach to use the variable values printed in logs to detect anomalies in large systems. Based on the variable values in logs, their approach creates invariants (e.g., equations). Any new logs that violates the invariant are considered to be a sign of anomalies. Fu \textsl{et al}$ . $~\cite{QFuanomaly} built a Finite State Automaton (FSA) using unstructured logs to detect performance bugs in distributed systems. 

\indent Xu \textsl{et al}$ . $~\cite{ConsoleLogs} link output logs to logs in source code to recover the text and and the variable parts of logs in source code. They applied Principal Component Analysis (PCA) to detect system anomalies. Tan \textsl{et al}$ . $~\cite{TanSalsa} propose a tool named SALSA, which constructs state-machines from logs. The state-machines are further used to detect anomalies in distributed computing platforms. Jiang \textsl{et al}$ . $~\cite{Jiang:2009:UCP:1525908.1525912} study the leverage of logs in troubleshooting issues from storage systems. They find that logs assist in a faster resolution of issues in storage systems. Beschastnikh \textsl{et al}$ . $~\cite{Beschastnikh:2011:LEI:2025113.2025151} designed automated tools that infers execution models from logs. These models can be used by developers to understand the behaviours of concurrent systems. Moreover, the models also assist in verifying the correctness of the system and fixing bugs.

To assist in fixing bugs using logs, Yuan \emph{et al$.$}~\cite{Yuan:2010:SED:1736020.1736038} propose an approach to automatically infer the failure scenarios when a log is printed during a failed run of a system.


Jiang \textsl{et al$ . $}~\cite{Jiang:2008:AAA:1400155.1400158,JiangICSM2008,JiangICSM20092,Jiang:2010:ICS:1850000.1850068} proposed log analysis approaches to assist in automatically verifying results from load tests. Their log analysis approaches first automatically abstract logs into system events~\cite{Jiang:2008:AAA:1400155.1400158}. Based on the such events, they identified both functional anomalies~\cite{JiangICSM2008} and performance degradations~\cite{JiangICSM20092} in load test results. In addition, they proposed an approach that leverage logs to reduce the load test that are performed in user environment~\cite{Jiang:2010:ICS:1850000.1850068}.

The extensive prior research of log analysis shows that logs are leveraged for different purposes and changes to logs can effect performance  of log analysis tools. This motivates our paper to study how much logs are changed by developers. As a first step, we study how many logs are stable and how many undergo changes across commits. Our findings show that more 20-80\% of the logs are changed at-least once and 3-11\% of the logs are changed frequently. 


\subsection{Empirical studies and tools developed on logs}


Prior research performs an empirical study on the characteristics of logs. Yuan \textsl{et al}$ . $~\cite{Characterizinglogs} studies the logging characteristics in four open source systems. They find that over 33\% of all log changes are after thoughts and logs are changed 1.8 times more regular entire code. Fu \textsl{et al$.$}~\cite{Fu1} performed an empirical study on where developers put logs. They find that logs are used for assertion checks, return value checks, exceptions, logic-branching and observing key points. The results of the analysis were evaluated by professionals from the industry and a F-score of over 95\% was achieved. 


Shang \textsl{et al$ . $}~\cite{IanGap} signify the fact that there is a gap between operators and developers of software systems, especially in the leverage of logs. They performed an empirical study on the evolution of both static logs and logs outputted during run time~\cite{EMSEIAN,PaperIanCIIII}. They find that logs are co-evolving with software systems. However, logs are often modified by developers without considering the needs of operators. Furthermore, Shang\textsl{ et al$ . $}~\cite{IanIcesm} find that understanding logs is challenging. They examine user mailing lists from three large open-source projects and find that users of these systems have various issues in understanding logs outputted by the system. Shang\textsl{ et al$ . $} propose to leverage different types of development knowledge, such as issue reports, to assist in understanding logs. 

The most related research by Yuan\textsl{ et al$ . $}~\cite{Yuan} shows that logs need to be improved by providing additional information. Their tool named \emph{Log Enhancer} can automatically provide additional control and data flow parameters into the logs thereby improving the logs. Our paper focuses more on informing developers which logs are more likely to get changed and identifying which factors explain the change of logs. 

% Log Advisor is another tool by Zhu \emph{et al$.$}~\cite{zhu2015learning} which helps in logging by learning where developers log through existing logging instances. 
%The most related prior research by Shang \emph{et al$.$}~\cite{EMSEIAN} empirically study the relationship of logging practice and code quality. Their manual analysis sheds light on the fact that some logs are changed due to field debugging. They also show that there is a strong relationship between logging practice and code quality. Our paper focused on understanding how logs are changed during bug fixes. Our results show that logs are leveraged extensively during bug fixes and may assist in the resolution of bugs. 





%on different systems to verify the results.


\section{Threats to Validity}
\label{threats}
In this section, we present the threats to the validity to our findings. \\


\noindent \textbf{External Validity}

Our case study is performed on Liferay, ActiveMQ, Camel and CloudStack. Though these studied systems have years of history and large user bases, these systems are all Java based platform softwares.  Software in other domains may not rely on logs and may not use log processing tools. Our projects are all open source and we do not verify the results on any commercial platform systems. More case studies on other domains and commercial platforms, with other programming languages are needed to see whether our findings can be generalized. 



\noindent \textbf{Construct Validity}


Our heuristics to extract logging source code may not be able to extract every log in the source code. Even though the studied systems-leverage logging libraries to generate logs at runtime, there may still exist user-defined logs. By manually examining the source code, we believe that we extract most of the logs. Evaluation on the coverage of our extracted logs can address this threat.


We use keywords to identify bug fixing commits when the JIRA issue ID is not included in the commit messages. Although such keywords are used extensively in prior research~\cite{EMSEIAN}, we may still fail to identify bug fixing commits or branching and merging commits.

We use Levenshtein ratio and choose a threshold to identify modifications to logs. However, such threshold may not accurately identify modifications to logs. Further sensitivity analysis on this threshold is needed to better understand the impact of the threshold to our findings.

Our models are incomplete, i.e., we have not measured all potential dimensions that impact logging stability like file and log ownership metrics, code complexity measures. However, we feel that our metrics are good and additional new dimensions can compliment our dimension to improve the explanation. 

\noindent \textbf{Internal Validity}

Our study is based on the data obtained from git and JIRA for all the studied systems. The quality of the data contained in the repositories can impact the internal validity of our study.

Our analysis of the relationship between metrics that are important factors in predicting the stability of logs cannot claim causal effects, as we are investigating correlation and not causation. The important factors from our random forest models only indicate that there exists a relationship which should be studied in depth in future studies. 





\section{Conclusion \cp{FIXME}}
\label{conc}
Logs are snippets of code, added by developers to record valuable information. The recorded information is used by a plethora of log processing tools to assist in software testing, monitoring performance and system state comprehension. These log processing tools are completely dependent on the logs and hence are affected when logs are changed.

 In this paper we study the stability of logs using a random forest classifier. The classifier is used to predict which logs are more likely to change in the future using context and log data.  The highlights of our study are:

\begin{itemize}
	\item We find that 35\%-50\% of logs are changed at-least once.
	\item Our random forest classifier for predicting whether a log will change achieves a precision of 89\%-91\% and recall of 71\%-83\%. 
	\item We find that log density, SLOC, developer experience, file ownership are important predictors of log stability in the studied applications.  	
%	\item We find that log density has negative correlation in all projects which suggests that when source code is well logged i.e., more logs per lines of code, the logs convey the needed information and are more stable. 
	
%	We find a negative correlation between developer experience and log changes from the developer dimension. We find that the number of comments in source code, also has negative correlation towards log changes. 
\end{itemize}

%Our findings highlight that we can predict which logs have a higher likelihood of getting changed when they are introduced. This information can be used by system administrators and practitioners, to identify logs which have a higher chance of affecting the log processing tools and prevent failure of these tools.  






%\begin{acknowledgements}
%If you'd like to thank anyone, place your comments here
%and remove the percent signs.
%\end{acknowledgements}

% BibTeX users please use one of
%\bibliographystyle{spbasic}      % basic style, author-year citations
%\bibliographystyle{spmpsci}      % mathematics and physical sciences
%\bibliographystyle{spphys}       % APS-like style for physics
%\bibliography{}   % name your BibTeX data base

% Non-BibTeX users please use
\bibliographystyle{IEEEtran}
\bibliography{Logreference}

\end{document}
% end of file template.tex

