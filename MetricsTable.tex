%% LyX 2.1.2 created this file.  For more info, see http://www.lyx.org/.
%% Do not edit unless you really know what you are doing.
\documentclass[english]{article}
\usepackage[T1]{fontenc}
\usepackage[latin9]{inputenc}
\usepackage{array}
\usepackage{multirow}

\makeatletter

%%%%%%%%%%%%%%%%%%%%%%%%%%%%%% LyX specific LaTeX commands.
%% Because html converters don't know tabularnewline
\providecommand{\tabularnewline}{\\}

\makeatother

\usepackage{babel}
\begin{document}
\begin{table}
\begin{raggedright}
\begin{tabular}{|c|c|c|>{\raggedright}p{0.6\columnwidth}|}
\hline 
Dimension & Metrics & Values & Definition (d) -- Rationale (r)\tabularnewline
\hline 
\hline 
\multirow{12}{*}{Code Metrics} & \multirow{2}{*}{Log revision count} & \multirow{2}{*}{Numeric } & d: The number of commits prior which had log statement changes. \tabularnewline
\cline{4-4} 
 &  &  & r: This helps to identify if the file is prune to log statement changes.\tabularnewline
\cline{2-4} 
 & \multirow{2}{*}{New File} & \multirow{2}{*}{Boolean (0 -1)} & d: Check if the log is added in a new file (i.e., newly comitted )\tabularnewline
\cline{4-4} 
 &  &  & r: This helps to identify which log statements where added later in
subsequent commits from the inital commit logs\tabularnewline
\cline{2-4} 
 & \multirow{2}{*}{Total Revision Count} & \multirow{2}{*}{Numeric} & d: Total number of commits made to the file since log statement is
added. \tabularnewline
\cline{4-4} 
 &  &  & r: This helps to find out if the file is changed heavily which can
result in log changes ( cite paper on after thoughts )\tabularnewline
\cline{2-4} 
 & \multirow{2}{*}{Code churn in commit} & \multirow{2}{*}{Numeric} & d: The code churn of the commit in which log is added. \tabularnewline
\cline{4-4} 
 &  &  & r: Log changes are correlated to code churn in files. ( Ian's paper
EMSE )\tabularnewline
\cline{2-4} 
 & \multirow{2}{*}{Variables declared} & \multirow{2}{*}{Numeric} & d: The number of variables which are declared before the log statement.
(we limit to 20 lines before log statement)\tabularnewline
\cline{4-4} 
 &  &  & r: When new variables are declared, developers may log the new variables
to obtain more information ( Afterthouhgs paper cite )\tabularnewline
\cline{2-4} 
 & \multirow{2}{*}{SLOC} & \multirow{2}{*}{Numeric} & d: The number of lines of code in the file.\tabularnewline
\cline{4-4} 
 &  &  & r: Large files have more functionality and are more prune to changes
( paper3 ) and more log changes. ( after thoughts and field debugging
EMSE)\tabularnewline
\hline 
\multirow{10}{*}{Log Metrics} & \multirow{2}{*}{Log Context} & \multirow{2}{*}{Categorical} & d: Identify the block in which log statement is added. (i.e., `if',
`if-else', `try-catch', `exception', `throw', `new function' )\tabularnewline
\cline{4-4} 
 &  &  & r: Prior research find that logs are used in assertion checks, logical
brancing, return value checking, assertion checking ( Where do developers
log )\tabularnewline
\cline{2-4} 
 & \multirow{2}{*}{Log Level} & \multirow{2}{*}{Categorical} & d: Identify the log level (verbosity) of the added log. ( i.e., `info',
`error', `warn', `debug', `trace' and `trace')\tabularnewline
\cline{4-4} 
 &  &  & r: Developers spend signifcant amount of time in adjusting the verboisty
of logging statemetns ( Characteriging logs )\tabularnewline
\cline{2-4} 
 & \multirow{2}{*}{Log variable count} & \multirow{2}{*}{Numerical} & d: Number of variables logged.\tabularnewline
\cline{4-4} 
 &  &  & r: Over 62\% of logging statemetns end adding new variables ( Characteriging
logs ) . Hence fewer variables in inital log statement might result
in addition later. \tabularnewline
\cline{2-4} 
 & \multirow{2}{*}{Log text length} & \multirow{2}{*}{Numerical} & d: Number of text phrases logged (i.e., we count all text present
between a pair of colons as one phrase )\tabularnewline
\cline{4-4} 
 &  &  & r: Over 45\% of logging statemetns have modifications to static context
(Characteriging logs). Logs with fewer phrases might be subject to
changes later to provide better explanation\tabularnewline
\cline{2-4} 
 & \multirow{2}{*}{Log density} & \multirow{2}{*}{Numerical} & d: Ratio of number of log lines to the source code lines in the file.\tabularnewline
\cline{4-4} 
 &  &  & r: Research has found that there is one log line per 30 lines of code.
(Charaterzing logs). If its less it suggests there may be additions
in later commits.\tabularnewline
\hline 
\end{tabular}
\par\end{raggedright}

\raggedright{}\protect\caption{}
\end{table}


\begin{table}
\begin{tabular}{|>{\centering}p{0.045\columnwidth}|c|c|>{\raggedright}p{0.6\columnwidth}|}
\hline 
Dimension & Metrics & Values & Definition (d) -- Rationale (r)\tabularnewline
\hline 
\multirow{12}{0.045\columnwidth}{Developer Metrics} & \multirow{2}{*}{Resolution time} & \multirow{2}{*}{Numerical} & d: The time it takes for the issue to get fixed. It is defined as
the time it takes since an issue is opened till its closed. \tabularnewline
\cline{4-4} 
 &  &  & r: More resolution time might suggest more complex fix with more code
churn resulting in more log churn. \tabularnewline
\cline{2-4} 
 & \multirow{2}{*}{Number of developers involved} & \multirow{2}{*}{Numeric} & d: Total number of unique developers who comment on the issue report
on JIRA\tabularnewline
\cline{4-4} 
 &  &  & r: Components with many unique authors likely lack strong ownership,
which in turn may lead to more defects (paper4) and change logging
statemetns (EMSE Ian). \tabularnewline
\cline{2-4} 
 & \multirow{2}{*}{Number of Comments} & \multirow{2}{*}{Numeric} & d: Total number of discussion posts on the issue. \tabularnewline
\cline{4-4} 
 &  &  & r: Number of comments is correlated to the resolution time of issue
reports (Predicting the fix time of bugs. In RSSE Giger). More comments
may also indicate the issue is more complex requiring more code churn
and logging statement changes. \tabularnewline
\cline{2-4} 
 & \multirow{2}{*}{Devloper experience} & \multirow{2}{*}{Numeric} & d: The number of commits the developer has made prior this commit. \tabularnewline
\cline{4-4} 
 &  &  & r: Research has shown that experienced developers might take up more
complex issues(Ownership, experience and defects: a fine-grained study
of authorship) and therefore may leverage logging statemetns more
(EMSE IAN). \tabularnewline
\cline{2-4} 
 & \multirow{2}{*}{Issue type} & \multirow{2}{*}{Categorical} & d: Identify the type of issue i.e., `Bug', `Improvement', `Task',
`New Feature', `Sub-Task', `Test' \tabularnewline
\cline{4-4} 
 &  &  & r: Some issue types might have higher code churn than others (example:
Bug and New features might have more code churn when compared to Sub-Tasks)
and are committed faster.\tabularnewline
\cline{2-4} 
 & \multirow{2}{*}{Priority type} & \multirow{2}{*}{Categorical} & d: Identigy the priority of the issue i.e., `Critical', `Blocker',
`Major', `Minor' and `Trivial'\tabularnewline
\cline{4-4} 
 &  &  & r: Research has shown that priority of issue effects resolution time
of bug fixes (Studying the Fix-Time for Bugs in Large Open Source
Projects). Higher the priorityindicates the issue will be fixed faster
with logging statement changes. \tabularnewline
\hline 
\end{tabular}\protect\caption{}
\end{table}

\end{document}
