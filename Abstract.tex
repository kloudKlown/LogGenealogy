%Logs are system generated outputs, created by logging statements in the code. Logs assist in understanding system behavior, monitoring choke-points and debugging. Prior research has demonstrated the importance of logs in operating, understanding and improving software systems. The importance of logs has lead to a new market of log management applications and tools. However, logs are often unstable i.e., being changed without the consideration of other stakeholders, causing misleading results and failure of log analysis tools. In order to pro actively mitigate such issues that are caused by unstable logs, in this paper we empirically study the stability of logs in four large software applications namely: Liferay, ActiveMQ, Camel and CloudStack. We find that 45\%-55\% of the logs are changed in these software applications. We build a random forest classifier to model if a log added to a file will be changed (i.e., effect the static text, variables or verbosity level of the logs) in the future, using context and log metrics calculated at the time of introduction of the log. Our classifier can model which logs will be changed in the future with 89\%-91\% precision and 71\%-83\% recall. We find that file ownership, developer experience, log density and SLOC are strong predictors of log stability in our models. Our findings can help develop more robust log processing tools and also help system administrators identify which logs are more likely to change and avoid depending on such unstable logs through critical analysis. 
Logs are created by logging statements and they	assist in understanding system behavior, monitoring choke-points and debugging. Prior research has demonstrated the importance of logging statements in operating, understanding and improving software systems. The importance of logs has lead to a new market of log management applications and tools. However, logs are often unstable, i.e., the logging statements that generate logs are often changed without the consideration of other stakeholders, causing misleading results and failure of log analysis tools. In order to proactively mitigate such issues that are caused by unstable logging statements, in this paper we empirically study the stability of logging statements in four open source applications namely: Liferay, ActiveMQ, Camel and CloudStack. We find that 20-45\% of the logging statements in our studied applications change throughout their lifetime. The median number of days between the introduction of a logging statement and the first change to that statement is between 1 and 17 in our studied applications. These numbers show that in order to reduce maintenance effort, developers of log processing tools must be careful when selecting the logging statements on which they will let their tools depend.

In this paper, we make an important first step towards assisting developers in deciding whether a logging statement is likely to remain unchanged in the future. Using random forest classifiers, we examine which metrics are important for understanding whether a logging statement will change. 
%We use metrics that are calculated from context and log information, to build a random forest classifier to understand which factors can increases the likelihood of a log change. 
We show that our classifiers achieve 83\%-91\% precision and 65\%-85\% recall in the four studied applications. We find that file ownership, developer experience, log density and SLOC are important metrics for deciding whether a logging statement will change in the future. Developers can use this knowledge to build more robust log processing tools, by making those tools depend on logs that are generated by logging statements which are likely to remain unchanged.

% By understanding which metrics are important for deciding whether a logging statement, practitioners can avoid the logs which have a higher likelihood of being changed and develop robust log processing tools.


%Our findings can help practitioners avoid depending on such unstable logs through critical analysis and develop more robust log processing tools

%which logs are more likely to change


% However, logs may change over time due to debugging, improvement or addition of new features and these changes have to be communicated to operators and administrators. In this paper, we study the different factors which can affect log stability. We conduct a case study on four large software applications namely: Liferay, ActiveMQ, Camel and CloudStack. We find that 45\%-55\% of the logs are changed in these software applications. We identify the log changes which effect the static text, variables or verbosity level of log and exclude other changes to logs. Next, we build models to predict if a log added to a file will change in the future. We use context and log  metrics calculated at the time of introduction of the log, to build a random forest classifier. Our classifier can predict which logs will change in the future with 89\%-91\% precision and 71\%-83\% recall. We find that file ownership, developer experience, log density and SLOC are strong predictors of log stability in our models and can help identify which logs are more likely to change in the future.On the one hand, this can help developers of log processing tools to develop more robust applications. On the other hand, system administrators can know before hand which logs are more likely to cause issues in the log processing tools, which can reduce their maintenance costs.


% We identify the four possible types of log changes namely 1) text modification, 2) variable modification, 3) log level change and 4) log relocations. We find that log relocation changes have no affect on log processing tools and we exclude these from our analysis.
 
 
% On the one hand, this can help developers of log processing tools to develop more robust applications. On the other hand, system administrators can know before hand which logs are more likely to cause issues in the log processing tools, which can reduce their maintenance costs.
