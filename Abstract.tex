Logs are system generated outputs, created from logging statements in the code. They help in understanding the system behavior, monitoring choke-points and in debugging. Prior research has demonstrated the importance of logs in operating, understanding and improving software systems which has lead to the development of log management applications and tools. However, logs may change over time due to debugging, improvement or addition of new features and these changes have to be communicated to operators and administrators. To understand the different factors which can affect log stability, in this paper we conduct a case study on four large software systems namely: Liferay, ActiveMQ, Camel and CloudStack. We first perform a manual analysis, where we find that 20-80\% of the logs are changed in these software systems. We identify the four possible types of log changes namely 1) text modification, 2) variable modification, 3) log level change and 4) log relocation and categorize all the log changes. Next, we build models to predict if a log added to a file will change in the future. We use change and product  metrics calculated at the time of introduction of the log, to build a random forest classifier. Our classifier can predict which logs will change in the future with precision of 89\%-91\% and recall of 71\%-91\%. We find that file ownership, developer experience, log density and SLOC are strong predictors of log stability in our models and can help identify which logs are more likely to change in the future.

% On the one hand, this can help developers of log processing tools to develop more robust applications. On the other hand, system administrators can know before hand which logs are more likely to cause issues in the log processing tools, which can reduce their maintenance costs.
