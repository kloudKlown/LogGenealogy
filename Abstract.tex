%Logs are system generated outputs, created by logging statements in the code. Logs assist in understanding system behavior, monitoring choke-points and debugging. Prior research has demonstrated the importance of logs in operating, understanding and improving software systems. The importance of logs has lead to a new market of log management applications and tools. However, logs are often unstable i.e., being changed without the consideration of other stakeholders, causing misleading results and failure of log analysis tools. In order to pro actively mitigate such issues that are caused by unstable logs, in this paper we empirically study the stability of logs in four large software applications namely: Liferay, ActiveMQ, Camel and CloudStack. We find that 45\%-55\% of the logs are changed in these software applications. We build a random forest classifier to model if a log added to a file will be changed (i.e., effect the static text, variables or verbosity level of the logs) in the future, using context and log metrics calculated at the time of introduction of the log. Our classifier can model which logs will be changed in the future with 89\%-91\% precision and 71\%-83\% recall. We find that file ownership, developer experience, log density and SLOC are strong predictors of log stability in our models. Our findings can help develop more robust log processing tools and also help system administrators identify which logs are more likely to change and avoid depending on such unstable logs through critical analysis. 

Logs are system generated outputs, created by logging statements in the code. Logs assist in understanding system behavior, monitoring choke-points and debugging. Prior research has demonstrated the importance of logs in operating, understanding and improving software systems. The importance of logs has lead to a new market of log management applications and tools. However, logs are often unstable i.e., being changed without the consideration of other stakeholders, causing misleading results and failure of log analysis tools. In order to proactively mitigate such issues that are caused by unstable logs, in this paper we empirically study the stability of logs in four large software applications namely: Liferay, ActiveMQ, Camel and CloudStack. We find that although around half of the logs are never changed, some logs are changed up to 10 times during development and more than half of the changed logs are changed within 7 commits after being added into the applications. 

We use metrics that are calculated from context and log information, to build a random forest classifier for predicting whether a log added to a file will be changed later. We show that our classifiers achieve 89\%-91\% precision, 71\%-83\% recall in the four studied applications. We find that file ownership, developer experience, log density and SLOC are important predictors of log stability in our models. Our findings can help practitioners avoid depending on such unstable logs through critical analysis and develop more robust log processing tools

%which logs are more likely to change


% However, logs may change over time due to debugging, improvement or addition of new features and these changes have to be communicated to operators and administrators. In this paper, we study the different factors which can affect log stability. We conduct a case study on four large software applications namely: Liferay, ActiveMQ, Camel and CloudStack. We find that 45\%-55\% of the logs are changed in these software applications. We identify the log changes which effect the static text, variables or verbosity level of log and exclude other changes to logs. Next, we build models to predict if a log added to a file will change in the future. We use context and log  metrics calculated at the time of introduction of the log, to build a random forest classifier. Our classifier can predict which logs will change in the future with 89\%-91\% precision and 71\%-83\% recall. We find that file ownership, developer experience, log density and SLOC are strong predictors of log stability in our models and can help identify which logs are more likely to change in the future.On the one hand, this can help developers of log processing tools to develop more robust applications. On the other hand, system administrators can know before hand which logs are more likely to cause issues in the log processing tools, which can reduce their maintenance costs.


% We identify the four possible types of log changes namely 1) text modification, 2) variable modification, 3) log level change and 4) log relocations. We find that log relocation changes have no affect on log processing tools and we exclude these from our analysis.
 
 
% On the one hand, this can help developers of log processing tools to develop more robust applications. On the other hand, system administrators can know before hand which logs are more likely to cause issues in the log processing tools, which can reduce their maintenance costs.
