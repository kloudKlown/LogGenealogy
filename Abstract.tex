Logs are system generated outputs, created from the logging statements in the code. They help in understanding the system behavior, monitoring choke-points and in debugging. Recent research has demonstrated the importance of logs in operating, understanding and improving software systems. This has lead to the development of log management applications and tools. However, logs may change over time due to debugging, improvement or addition of new features. When logs are changed, the changes have to be communicated to operators and administrators. In addition, the log processing and management applications have to be updated. 

In this paper we study the stability of logs on four large software systems namely: Liferay, ActiveMQ, Camel and CloudStack. We conduct a manual analysis and find that 20-80\% of the logs are changed in these software systems. We identify the four different types of log changes namely 1) text modification, 2) variable modification, 3) log level change and 4) log relocation. After characterizing the different types of log changes we build models to predict if a log added to a file will change in the future. We use code churn, log churn and developer metrics to build a random forest classifier. Our classifier can predict which logs will change in the future with 89-93 \% precision, and 76-92\% recall.  We find that code, log and developer metrics are strong predictors of log stability in our models and can help identify which logs are more likely to change in the future. On the one hand, this can help developers of log processing tools to develop more robust applications. On the other hand, system administrators can know before hand which logs are more likely to cause issues in the log processing tools, which can reduce their maintenance costs.
