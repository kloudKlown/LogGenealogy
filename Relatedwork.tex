
In this section, we present prior research that performs log analysis on large software systems and tools developed to assist in logging. 


\subsection{Log Analysis}

% The purpose of this section is to highlight the research that shows
%logs are used in debugging process. This is related to our work because
%(1) we show that logs are used in all the processes of a software
%life cycle, (2) how logs are useful in the improving the quality of
%the software. 
%\ian{some names are textsl, some are not. I think just keep the name as regular font and put et al. as italic, sometimes you miss the . in et al. and you should put dollar signs before and after the .}
Prior work leverages logs for testing and detecting anomalies in large scale systems. Shang\textsl{ et al$.$}~\cite{IanContextinformation} propose an approach to leverage logs in verifying the deployment of Big Data Analytics applications. Their approach analyzes logs in order to find differences between running in a small testing environment and a large field environment. Lou \textsl{et al}$.$~\cite{JGLouMining} propose an approach to use the variable values printed in logs to detect anomalies in large systems. Based on the variable values in logs, their approach creates invariants (e.g., equations). Any new logs that violates the invariant are considered to be a sign of anomalies. Fu \textsl{et al}$ . $~\cite{QFuanomaly} built a Finite State Automaton (FSA) using unstructured logs to detect performance bugs in distributed systems. 

\indent Xu \textsl{et al}$ . $~\cite{ConsoleLogs} link output logs to logs in source code to recover the text and and the variable parts of logs in source code. They applied Principal Component Analysis (PCA) to detect system anomalies. Tan \textsl{et al}$ . $~\cite{TanSalsa} propose a tool named SALSA, which constructs state-machines from logs. The state-machines are further used to detect anomalies in distributed computing platforms. Jiang \textsl{et al}$ . $~\cite{Jiang:2009:UCP:1525908.1525912} study the leverage of logs in troubleshooting issues from storage systems. They find that logs assist in a faster resolution of issues in storage systems. Beschastnikh \textsl{et al}$ . $~\cite{Beschastnikh:2011:LEI:2025113.2025151} designed automated tools that infers execution models from logs. These models can be used by developers to understand the behaviours of concurrent systems. Moreover, the models also assist in verifying the correctness of the system and fixing bugs.

To assist in fixing bugs using logs, Yuan \emph{et al$.$}~\cite{Yuan:2010:SED:1736020.1736038} propose an approach to automatically infer the failure scenarios when a log is printed during a failed run of a system.


Jiang \textsl{et al$ . $}~\cite{Jiang:2008:AAA:1400155.1400158,JiangICSM2008,JiangICSM20092,Jiang:2010:ICS:1850000.1850068} proposed log analysis approaches to assist in automatically verifying results from load tests. Their log analysis approaches first automatically abstract logs into system events~\cite{Jiang:2008:AAA:1400155.1400158}. Based on the such events, they identified both functional anomalies~\cite{JiangICSM2008} and performance degradations~\cite{JiangICSM20092} in load test results. In addition, they proposed an approach that leverage logs to reduce the load test that are performed in user environment~\cite{Jiang:2010:ICS:1850000.1850068}.

The extensive prior research of log analysis shows that logs are leveraged for different purposes and changes to logs can effect performance  of log analysis tools. This motivates our paper to study how much logs are changed by developers. As a first step, we study how many logs are stable and how many undergo changes across commits. Our findings show that more 20-80\% of the logs are changed at-least once and 3-11\% of the logs are changed frequently. 


\subsection{Empirical studies and tools developed on logs}


Prior research performs an empirical study on the characteristics of logs. Yuan \textsl{et al}$ . $~\cite{Characterizinglogs} studies the logging characteristics in four open source systems. They find that over 33\% of all log changes are after thoughts and logs are changed 1.8 times more regular entire code. Fu \textsl{et al$.$}~\cite{Fu1} performed an empirical study on where developers put logs. They find that logs are used for assertion checks, return value checks, exceptions, logic-branching and observing key points. The results of the analysis were evaluated by professionals from the industry and a F-score of over 95\% was achieved. 


Shang \textsl{et al$ . $}~\cite{IanGap} signify the fact that there is a gap between operators and developers of software systems, especially in the leverage of logs. They performed an empirical study on the evolution of both static logs and logs outputted during run time~\cite{EMSEIAN,PaperIanCIIII}. They find that logs are co-evolving with software systems. However, logs are often modified by developers without considering the needs of operators. Furthermore, Shang\textsl{ et al$ . $}~\cite{IanIcesm} find that understanding logs is challenging. They examine user mailing lists from three large open-source projects and find that users of these systems have various issues in understanding logs outputted by the system. Shang\textsl{ et al$ . $} propose to leverage different types of development knowledge, such as issue reports, to assist in understanding logs. 

The most related research by Yuan\textsl{ et al$ . $}~\cite{Yuan} shows that logs need to be improved by providing additional information. Their tool named \emph{Log Enhancer} can automatically provide additional control and data flow parameters into the logs thereby improving the logs. Our paper focuses more on informing developers which logs are more likely to get changed and identifying which factors explain the change of logs. 

% Log Advisor is another tool by Zhu \emph{et al$.$}~\cite{zhu2015learning} which helps in logging by learning where developers log through existing logging instances. 
%The most related prior research by Shang \emph{et al$.$}~\cite{EMSEIAN} empirically study the relationship of logging practice and code quality. Their manual analysis sheds light on the fact that some logs are changed due to field debugging. They also show that there is a strong relationship between logging practice and code quality. Our paper focused on understanding how logs are changed during bug fixes. Our results show that logs are leveraged extensively during bug fixes and may assist in the resolution of bugs. 





%on different systems to verify the results.
