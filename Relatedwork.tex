
In this section, we present prior research in which log behavior in software systems is analyzed. In addition, we discuss tools developed to assist in logging. 

\subsection{Log Analysis}

Prior work leverages logs for detecting anomalies in large scale systems. Lou \textsl{et al}$.$~\cite{JGLouMining} propose an approach to use the variable values printed in logs to detect anomalies in large systems. Based on the variable values in logs, their approach creates invariants (e.g., equations). Any new logs that violates the invariant are considered to be a sign of anomalies. Fu \textsl{et al}$ . $~\cite{QFuanomaly} built a Finite State Automaton (FSA) using unstructured logs to detect performance bugs in distributed systems.  \indent Xu \textsl{et al}$.$~\cite{ConsoleLogs} link output logs to logs in source code to recover the text and and the variable parts of logs in source code. They applied Principal Component Analysis (PCA) to detect system anomalies. To assist in fixing bugs using logs, Yuan \emph{et al$.$}~\cite{Yuan:2010:SED:1736020.1736038} propose an approach to automatically infer the failure scenarios when a log is printed during a failed run of a system.


Logs are leveraged during load testing of large scale systems. The data collected from logs during load tests helps developers diagnose faults in the system. Jiang \textsl{et al$ . $}~\cite{Jiang:2008:AAA:1400155.1400158,JiangICSM2008,JiangICSM20092,Jiang:2010:ICS:1850000.1850068} proposed log analysis approaches to assist in automatically verifying results from load tests. Their log analysis approaches first automatically abstract logs into system events~\cite{Jiang:2008:AAA:1400155.1400158}. Based on the such events, they identified both functional anomalies~\cite{JiangICSM2008} and performance degradations~\cite{JiangICSM20092} in load test results. In addition, they proposed an approach that leverage logs to reduce the load test that are performed in user environment~\cite{Jiang:2010:ICS:1850000.1850068}. Shang\textsl{ et al$.$}~\cite{IanContextinformation} propose an approach to leverage logs in verifying the deployment of Big Data Analytics applications. Their approach analyzes logs in order to find differences between running in a small testing environment and a large field environment.


Logs are also leveraged in debugging software systems. Jiang \textsl{et al}$ . $~\cite{Jiang:2009:UCP:1525908.1525912} study the leverage of logs in troubleshooting issues from storage systems. They find that logs assist in a faster resolution of issues in storage systems. Beschastnikh \textsl{et al}$ . $~\cite{Beschastnikh:2011:LEI:2025113.2025151} designed automated tools that infers execution models from logs. These models can be used by developers to understand the behaviors of concurrent systems. Moreover, the models also assist in verifying the correctness of the system and fixing bugs.


The extensive prior research of log analysis shows that logs are leveraged for different purposes and changes to logs can effect performance  of log analysis tools. This motivates our paper to study how much logs are changed by developers. As a first step, we study how many logs are stable and how many undergo changes across commits. Our findings show that more 20-80\% of the logs are changed at-least once and 3-11\% of the logs are changed frequently. 


\subsection{Empirical Studies on Logs}


Prior research performs an empirical study on the characteristics of logs. Yuan \textsl{et al}$ . $~\cite{Characterizinglogs} studies the logging characteristics in four open source systems. They find that over 33\% of all log changes are after-thoughts and logs are changed 1.8 times more than regular code. Fu \textsl{et al$.$}~\cite{Fu1} performed an empirical study on where developers put logs. They find that logs are used for assertion checks, return value checks, exceptions, logic-branching and observing key points. The results of the analysis were evaluated by professionals from the industry and a F-score of over 95\% was achieved. 

Tan \textsl{et al}$ . $~\cite{TanSalsa} propose a tool named SALSA, which constructs state-machines from logs. The state-machines are further used to detect anomalies in distributed computing platforms. Yuan\textsl{ et al$ . $}~\cite{Yuan} shows that logs need to be improved by providing additional . Their tool named \emph{Log Enhancer} can automatically provide additional control and data flow parameters into the logs thereby improving the logs. While these works focus more on enhancing existing logs in the system, our paper focuses more on informing developers which logs are more likely to get changed and identifying which factors explain the change of logs. 


Research also shows that logs are source of information about the execution of large software systems for developers and end users. Shang \textsl{et al$ . $} performed an empirical study on the evolution of both static logs and logs outputted during run time~\cite{EMSEIAN,PaperIanCIIII}. They find that logs are co-evolving with software systems. However, logs are often modified by developers without considering the needs of operators which even effects the log processing tools which run on top of them. They highlight the fact that there is a gap between operators and developers of software systems, especially in the leverage of logs~\cite{IanGap}. Furthermore, Shang\textsl{ et al$ . $}~\cite{IanIcesm} find that understanding logs is challenging. They examine user mailing lists from three large open-source projects and find that users of these systems have various issues in understanding logs outputted by the system. These research works highlight that developers and system operators leverage logs and changing logs can effect both. Our work tries to tackle this problem by identifying logs which are more likely to change in the early stages so developers and system operators so they can be made more stable.  
