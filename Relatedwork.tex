

We present prior research that is related to this paper. 

%in which log behavior in software applications is analyzed. In addition, we discuss tools developed to assist in logging. 

\subsection{Log Maintenance Tools}


%Much of the prior work leverages logs for detecting anomalies in large scale systems. Lou {et al}$.$~\cite{JGLouMining} propose an approach to use the variable values printed in logs to detect anomalies in large systems. Similarly, Fu {et al}$ . $~\cite{QFuanomaly} built a finite state automaton using unstructured logs to detect performance bugs in distributed systems. In debugging, Yuan {et al$.$}~\cite{Yuan:2010:SED:1736020.1736038} propose an approach to automatically infer the failure scenarios when a log is printed during a failed run of a system.
%
%
%Logs are leveraged during load testing of large scale systems. The data collected from logs during load tests helps developers diagnose faults in the system. Jiang {et al$ . $}~\cite{Jiang:2008:AAA:1400155.1400158,JiangICSM2008,JiangICSM20092,Jiang:2010:ICS:1850000.1850068} proposed log analysis approaches to assist in automatically verifying results from load tests. Their log analysis approaches first automatically abstract logs into system events~\cite{Jiang:2008:AAA:1400155.1400158}. Based on the such events, they identified both functional anomalies~\cite{JiangICSM2008} and performance degradations~\cite{JiangICSM20092} in load test results.
%\ian{This subsection should be about assisting developers in log maintenance. SALSA is not related. In stead, you should cite where to log in icse 2014 SEIP track and ICSE 2015.}

%<<<<<<< HEAD

%Tan {et al}$ . $~\cite{TanSalsa} propose a tool named SALSA, which constructs state-machines from logs. The state-machines are further used to detect anomalies in distributed computing platforms.
%Prior research performs an empirical study on the characteristics of logging statements. 
%Yuan et al.~\cite{Characterizinglogs} study the logging characteristics in four open source systems. They find that over 33\% of all changes to logging statements are after-thoughts and that logging statements are changed 1.8 times more often than regular code.  


Prior research has explored various approaches in order to assist developers in maintaining logs. Research by Fu {et al$.$}~\cite{Fu1} explores where developers put logging statements in their code and provides guidelines for more effective logging. A recent work by Zhu {et al$.$}~\cite{ZhuIcse15} helps developer log effectively during development and provides a suggestive tool named \emph{Log Advisor}, to assist in logging. Yuan {et al $.$}~\cite{Yuan_beconservative:} show that logs can be effectively used to diagnose system failures and provide a tool named \emph{Errlog}, to proactively add logging statements. A follow-up work done by them Yuan {et al$ . $}~\cite{Yuan} show that logs need to be improved by providing additional information. Their tool named \emph{Log Enhancer} can automatically provide additional control and data flow parameters into the logs thereby improving the logs. \emph{Log Enhancer} can improve the quality of log added and mitigate the need for changes later. Though prior research tries to place the most stable logging statements in software, they do not provide any insight into why some logging statements are more likely to be changed. Our paper tries to determine which logging statements have higher likelihood of being changed and avoid using such logging statements in the log processing tools. 

% from the analysis.

%changed later and avoid such logging statements from the analysis. 

%They find that logging statements are used for assertion checks, return value checks, exceptions, logic-branching and observing key points. The results of the analysis were evaluated by professionals from the industry and an F-measure of over 95\% was achieved.



%=======
%et al}$ . $~\cite{TanSalsa} propose a tool named SALSA, which constructs state-machines from logs. The state-machines are further used to detect anomalies in distributed computing platforms.\ian{why the first paper is even discussed here?} Yuan {et al$ . $}~\cite{Yuan} show that logs need to be improved by providing additional\ian{you missed a word here?} . Their tool named \emph{Log Enhancer}\ian{in intro, you name this tool will all lower case letters} can automatically provide additional control and data flow parameters into the logs thereby improving the logs. \emph{Log Enhancer} can improve the quality of log added and mitigate the need for changes later. However, it does not provide any insight into why some logging statements are more likely to be changed. Our paper, tries to classify which logging statements have higher likelihood of being changed later and avoid such logging statements from the analysis. 


%tools assist developers to improve logging so developers can log the right information when the log is introduced


% show that output logs are leveraged for various activities such as debugging, anomaly detection, load test analysis and diagnose faults. However, when logging statements are changed, the output logs leveraged by these tools and analysis can change. These changes in the output logs, can break the tools and leads to erroneous results in the analysis.

% Based on the variable values in logs, their approach creates invariants (e.g., equations). Any new logs that violates the invariant are considered to be a sign of anomalies. Fu {et al}$ . $~\cite{QFuanomaly} built a finite state automaton using unstructured logs to detect performance bugs in distributed systems.  \indent Xu {et al}$.$~\cite{ConsoleLogs} link output logs to logs in source code to recover the text and and the variable parts of logs in source code. They applied Principal Component Analysis (PCA) to detect system anomalies. To assist in fixing bugs using logs, Yuan {et al$.$}~\cite{Yuan:2010:SED:1736020.1736038} propose an approach to automatically infer the failure scenarios when a log is printed during a failed run of a system.
%
%

%Logs are leveraged during load testing of large scale systems. The data collected from logs during load tests helps developers diagnose faults in the system. Jiang {et al$ . $}~\cite{Jiang:2008:AAA:1400155.1400158,JiangICSM2008,JiangICSM20092,Jiang:2010:ICS:1850000.1850068} proposed log analysis approaches to assist in automatically verifying results from load tests. Their log analysis approaches first automatically abstract logs into system events~\cite{Jiang:2008:AAA:1400155.1400158}. Based on the such events, they identified both functional anomalies~\cite{JiangICSM2008} and performance degradations~\cite{JiangICSM20092} in load test results. The extensive prior research on log analysis shows that logs are leveraged for different purposes and changing logs can affect the performance of log analysis tools. 


%and changes to logs can affect performance  of log analysis tools. 

% In addition, they proposed an approach that leverage logs to reduce the load test that are performed in user environment~\cite{Jiang:2010:ICS:1850000.1850068}. Shang {et al$.$}~\cite{IanContextinformation} propose an approach to leverage logs in verifying the deployment of Big Data Analytics applications. Their approach analyzes logs in order to find differences between running in a small testing environment and a large field environment.
%
%
%Logs are also leveraged in debugging software systems. Jiang {et al}$.$~\cite{Jiang:2009:UCP:1525908.1525912} study the leverage of logs in troubleshooting issues from storage systems. They find that logs assist in a faster resolution of issues in storage systems. Beschastnikh \textsl{et al}$ . $~\cite{Beschastnikh:2011:LEI:2025113.2025151} designed automated tools that infers execution models from logs. These models can be used by developers to understand the behaviors of concurrent systems. Moreover, the models also assist in verifying the correctness of the system and fixing bugs.
%\subsection {Log Tools}


%While these works focus more on enhancing existing logs in the system, our paper focuses more on informing developers which logs are more likely to get changed and identifying which factors explain the change of logs. 


%As a first step, we study how many logs are stable and how many undergo changes across commits. Our findings show that more 20-80\% of the logs are changed at-least once and 3-11\% of the logs are changed frequently. 


\subsection{Empirical Studies on Logging Statements}


%empirical studies on logs, show that 1) logging statements are changed by developers for various reasons 2) changes to logging stateme


%and there are many contributing factors for the change. This motivates our study as we find that logs are changed for different reasons. It also helps in 

%In our work, we use the intuitions and characteristics to collect relevant context and log metrics, to correctly classify the likelihood of log change.   
Prior research performs empirical studies to understand the characteristics of logging statements. 
Yuan et al.~\cite{Characterizinglogs} study the logging characteristics in four open source systems and finds that logging statements are changed 1.8 times more than regular code. Shang et al. performed an empirical study on the evolution of both static logs and logs outputted during run time~\cite{EMSEIAN,PaperIanCIIII}. They find that logging statements are co-evolving with software systems. However, logging statements are often modified by developers without considering the needs of operators which even affects the log processing tools which run on top of the logs produced by these statements. Shang highlight the fact that there is a gap between operators and developers of software systems, especially in the leverage of logs~\cite{IanGap}. Furthermore, Shang et al.~\cite{IanIcesm} find that understanding logs is challenging. They examine user mailing lists from three large open-source projects and find that users of these systems have various issues in understanding logs outputted by the system. 


The existing empirical studies on logging statements show that 1) logs are leveraged by developers for different purposes and 2) logging statements are changed extensively by developers without consideration of other stakeholders, which affect practitioners and end users. These findings highlight the need for better understanding of the factors determining the likelihood of a logging statement changing. 

%motivate the need for understanding the stability of logging statements in applications, which is not done by any prior studies to our best knowledge. 



  
% This gap between developers and system operators can be avoided if the logging statements leverage 

% research works highlight that developers and system operators leverage logs and changing logging statements can affect both.

%. Our work tries to tackle this problem by identifying logs which are more likely to change in the early stages so developers and system operators so they can be made more stable.  
