Logging statements are snippets of code, added by developers to yield valuable information about the execution of an application. Logging statements generate their output in logs, which are used by a plethora of log processing tools to assist in software testing, performance monitoring and system state comprehension. These log processing tools are completely dependent on the logs and hence are affected when logging statements are changed.

In order to reduce the effort required for the maintenance of log processing tools, we examine changes made to logging statements in four open source projects. The goal of our work is to help developers of log processing tools decide whether a logging statement is likely to change in the future. We consider our work an important first step towards helping developers to build more robust log processing tools, as knowing whether a log will change in the future allows developers to let their log processing tools rely on logs generated by logging statements that are likely to remain unchanged. The highlights of our work are:

%In this paper we study the stability of logs using a random forest classifier. The classifier is used to predict which logs are more likely to change in the future using context and log data.  The highlights of our study are:

\begin{itemize}
	\item We find that 12\%-40\%  of logs are changed at-least once.
	\item Our random forest classifier for predicting whether a log will change achieves a precision of 83\%-91\% and recall of 65\%-85\%. 
	\item We find that log density, SLOC, developer experience, file ownership are important predictors of log stability in the studied applications.  	
%	\item We find that log density has negative correlation in all projects which suggests that when source code is well logged i.e., more logs per lines of code, the logs convey the needed information and are more stable. 
	
%	We find a negative correlation between developer experience and log changes from the developer dimension. We find that the number of comments in source code, also has negative correlation towards log changes. 
\end{itemize}

Our findings highlight that we can correctly classify the likelihood of a log change, when log is added into the application. The important metrics from the classifier help in determining the likelihood of a log change, and developers can use this knowledge to be more selective when importing logging statements into their processing tools. 

% which logs have a higher likelihood of getting changed when they are introduced. This information can be used by system administrators and practitioners, to identify logs which have a higher chance of affecting the log processing tools and prevent failure of these tools.  



