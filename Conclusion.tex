Logs are snippets of code, introduced by experts to record valuable information. This information is used by a plethora of log processing tools to assist in software testing, monitoring performance and system state comprehension. These log processing tools are completely dependent on the logs and hence are impacted when logs are changed.

 In this paper we study the stability of logs, which affects the performance of the processing tools which utilize them. We build a random f orest classifier to predict which logs are more likely to change in the future using data from three dimensions namely: \emph{code, log} and \emph{developers}. We collect this data when logs are added in the source code and we predict which log will change in the future using this data. The highlights of our study are:

\begin{itemize}
	\item We find that 20-80 \% of logs are changed at-least once.
	\item Our random forest classifier for predicting whether a log will change achieves a precision of 89-93\% and recall of 76-92\%. 
	\item We find that logs written by more experienced developers are less likely to change in the future and are more stable. We also find that logs added in commit with more \# comments in JIRA report are less likely to change in future.
	\item We find increase in the number of variables declared in commit increases the likelihood of log change. We find similar positive correlation to log change in SLOC, log density and log text length. This suggests that logs present in larger files with more context information are more likely to be changed in the future.
	
%	We find a negative correlation between developer experience and log changes from the developer dimension. We find that the number of comments in source code, also has negative correlation towards log changes. 
\end{itemize}

Our findings highlight that we can predict which logs have a higher likelihood of getting changed when they are introduced. This information can be used by system administrators and practitioners, to identify logs which have a higher chance of affecting the log processing tools and prevent failure of these tools.  



